% Options for packages loaded elsewhere
\PassOptionsToPackage{unicode}{hyperref}
\PassOptionsToPackage{hyphens}{url}
%
\documentclass[
  twoside,
  12pt,
  letterpaper,
  fleqn]{article}

\usepackage{amsmath,amssymb}
\usepackage{lmodern}
\usepackage{iftex}
\ifPDFTeX
  \usepackage[T1]{fontenc}
  \usepackage[utf8]{inputenc}
  \usepackage{textcomp} % provide euro and other symbols
\else % if luatex or xetex
  \usepackage{unicode-math}
  \defaultfontfeatures{Scale=MatchLowercase}
  \defaultfontfeatures[\rmfamily]{Ligatures=TeX,Scale=1}
\fi
% Use upquote if available, for straight quotes in verbatim environments
\IfFileExists{upquote.sty}{\usepackage{upquote}}{}
\IfFileExists{microtype.sty}{% use microtype if available
  \usepackage[]{microtype}
  \UseMicrotypeSet[protrusion]{basicmath} % disable protrusion for tt fonts
}{}
\makeatletter
\@ifundefined{KOMAClassName}{% if non-KOMA class
  \IfFileExists{parskip.sty}{%
    \usepackage{parskip}
  }{% else
    \setlength{\parindent}{0pt}
    \setlength{\parskip}{6pt plus 2pt minus 1pt}}
}{% if KOMA class
  \KOMAoptions{parskip=half}}
\makeatother
\usepackage{xcolor}
\setlength{\emergencystretch}{3em} % prevent overfull lines
\setcounter{secnumdepth}{5}
% Make \paragraph and \subparagraph free-standing
\ifx\paragraph\undefined\else
  \let\oldparagraph\paragraph
  \renewcommand{\paragraph}[1]{\oldparagraph{#1}\mbox{}}
\fi
\ifx\subparagraph\undefined\else
  \let\oldsubparagraph\subparagraph
  \renewcommand{\subparagraph}[1]{\oldsubparagraph{#1}\mbox{}}
\fi


\providecommand{\tightlist}{%
  \setlength{\itemsep}{0pt}\setlength{\parskip}{0pt}}\usepackage{longtable,booktabs,array}
\usepackage{calc} % for calculating minipage widths
% Correct order of tables after \paragraph or \subparagraph
\usepackage{etoolbox}
\makeatletter
\patchcmd\longtable{\par}{\if@noskipsec\mbox{}\fi\par}{}{}
\makeatother
% Allow footnotes in longtable head/foot
\IfFileExists{footnotehyper.sty}{\usepackage{footnotehyper}}{\usepackage{footnote}}
\makesavenoteenv{longtable}
\usepackage{graphicx}
\makeatletter
\def\maxwidth{\ifdim\Gin@nat@width>\linewidth\linewidth\else\Gin@nat@width\fi}
\def\maxheight{\ifdim\Gin@nat@height>\textheight\textheight\else\Gin@nat@height\fi}
\makeatother
% Scale images if necessary, so that they will not overflow the page
% margins by default, and it is still possible to overwrite the defaults
% using explicit options in \includegraphics[width, height, ...]{}
\setkeys{Gin}{width=\maxwidth,height=\maxheight,keepaspectratio}
% Set default figure placement to htbp
\makeatletter
\def\fps@figure{htbp}
\makeatother


\usepackage{geometry}

\geometry{reset, letterpaper, height=9in, width=6in, hmarginratio=1:1, vmarginratio=1:1, marginparsep=0pt, marginparwidth=0pt, headheight=15pt}

\usepackage{lipsum}

\def\dklogo{\raisebox{-.2\height}{\includegraphics[width=12pt]{../../assets/direct-knowledge-logo-book.png}}\footnotesize{\ Direct Knowledge}}
\usepackage{fancyhdr}
\pagestyle{fancy}
\fancyhead[LE]{\thepage}%
\fancyhead[RE]{\dklogo}%
\fancyhead[CE]{}%
\fancyfoot[LE]{}%
\fancyfoot[RE]{}%
\fancyfoot[CE]{}%
\fancyhead[LO]{D. A. Smith}%
\fancyhead[RO]{\thepage}%
\fancyhead[CO]{}%
\fancyfoot[LO]{}%
\fancyfoot[RO]{}%
\fancyfoot[CO]{}%


\renewcommand{\headrulewidth}{0.4pt}
\renewcommand{\footrulewidth}{0pt}

\usepackage{amssymb}
\usepackage{amsmath}
\usepackage{amsthm}
\usepackage{graphicx} 


\usepackage[most]{tcolorbox} 
\definecolor{block-gray}{gray}{0.97}
\newtcolorbox{zitat}[1][]{%
    colback=block-gray,
    grow to right by=-10mm,
    grow to left by=-10mm, 
    boxrule=0pt,
    boxsep=0pt,
    breakable,
    enhanced jigsaw,
    borderline west={2pt}{0pt}{gray},
    colbacktitle={block-gray},
    coltitle={black},
    fonttitle={\large\bfseries},
    attach title to upper={},
    #1,
}
\renewcommand{\quote}{\zitat}
\renewcommand{\endquote}{\endzitat}

\def\banner{ {\rule{\linewidth}{0.2pt}}
    \begin{minipage}[c]{14px}\includegraphics[width=14pt]{../../assets/direct-knowledge-logo-book.png}\end{minipage}
    \begin{minipage}[c]{150px}Direct Knowledge | Tutorials \end{minipage}
}

\usepackage{titling}
\setlength{\droptitle}{-8ex}
\pretitle{\vspace{-20pt}\banner\vspace{10pt}\begin{flushleft}\huge\bfseries}
\posttitle{\par\end{flushleft}}
\preauthor{\begin{flushleft}}
\postauthor{,
    \small{Direct Knowledge, USA}
    \footnote{email: david@directknowledge.com}
    \footnote{\copyright \, 2023 \ David A. Smith}
    \footnote{With authorization, you can freely share and reproduce portions of this work for educational or personal use. Please note that distributing any portion of it in print form requires further permission from its original authors, as does posting online to public servers or mailing lists without prior consent.}
    \end{flushleft}
    }
\predate{\begin{flushleft}}
\postdate{\end{flushleft}}

\renewenvironment{abstract}
{\par\noindent\textbf{\abstractname.}\ \ignorespaces \itshape}
{\par\medskip}


\usepackage{setspace}
\linespread{1.25}
\AtEndEnvironment{solution}{\vspace{-1.5\baselineskip}\hfill\qedsymbol}
\AtEndEnvironment{proof}{\vspace{-1.25\baselineskip}}




\makeatletter
\makeatother
\makeatletter
\makeatother
\makeatletter
\@ifpackageloaded{caption}{}{\usepackage{caption}}
\AtBeginDocument{%
\ifdefined\contentsname
  \renewcommand*\contentsname{Table of contents}
\else
  \newcommand\contentsname{Table of contents}
\fi
\ifdefined\listfigurename
  \renewcommand*\listfigurename{List of Figures}
\else
  \newcommand\listfigurename{List of Figures}
\fi
\ifdefined\listtablename
  \renewcommand*\listtablename{List of Tables}
\else
  \newcommand\listtablename{List of Tables}
\fi
\ifdefined\figurename
  \renewcommand*\figurename{Figure}
\else
  \newcommand\figurename{Figure}
\fi
\ifdefined\tablename
  \renewcommand*\tablename{Table}
\else
  \newcommand\tablename{Table}
\fi
}
\@ifpackageloaded{float}{}{\usepackage{float}}
\floatstyle{ruled}
\@ifundefined{c@chapter}{\newfloat{codelisting}{h}{lop}}{\newfloat{codelisting}{h}{lop}[chapter]}
\floatname{codelisting}{Listing}
\newcommand*\listoflistings{\listof{codelisting}{List of Listings}}
\usepackage{amsthm}
\theoremstyle{definition}
\newtheorem{definition}{Definition}[section]
\theoremstyle{definition}
\newtheorem{example}{Example}[section]
\theoremstyle{plain}
\newtheorem{proposition}{Proposition}[section]
\theoremstyle{plain}
\newtheorem{theorem}{Theorem}[section]
\theoremstyle{remark}
\AtBeginDocument{\renewcommand*{\proofname}{Proof}}
\newtheorem*{remark}{Remark}
\newtheorem*{solution}{Solution}
\makeatother
\makeatletter
\@ifpackageloaded{caption}{}{\usepackage{caption}}
\@ifpackageloaded{subcaption}{}{\usepackage{subcaption}}
\makeatother
\makeatletter
\@ifpackageloaded{tcolorbox}{}{\usepackage[many]{tcolorbox}}
\makeatother
\makeatletter
\@ifundefined{shadecolor}{\definecolor{shadecolor}{rgb}{.97, .97, .97}}
\makeatother
\makeatletter
\makeatother
\ifLuaTeX
  \usepackage{selnolig}  % disable illegal ligatures
\fi
\usepackage[citestyle = authoryear]{biblatex}
\addbibresource{../references.bib}
\IfFileExists{bookmark.sty}{\usepackage{bookmark}}{\usepackage{hyperref}}
\IfFileExists{xurl.sty}{\usepackage{xurl}}{} % add URL line breaks if available
\urlstyle{same} % disable monospaced font for URLs
\hypersetup{
  pdftitle={Incidence Geometry},
  pdfauthor={David A. Smith},
  pdfkeywords={incidence geometry, point, line, incidence, geometry},
  hidelinks,
  pdfcreator={LaTeX via pandoc}}

\title{Incidence Geometry}
\usepackage{etoolbox}
\makeatletter
\providecommand{\subtitle}[1]{% add subtitle to \maketitle
  \apptocmd{\@title}{\par {\large #1 \par}}{}{}
}
\makeatother
\subtitle{and How to Write Rigorous Proofs}
\author{David A. Smith}
\date{Thursday, February 9, 2023}

\begin{document}
\maketitle
\ifdefined\Shaded\renewenvironment{Shaded}{\begin{tcolorbox}[breakable, frame hidden, enhanced, boxrule=0pt, borderline west={3pt}{0pt}{shadecolor}, interior hidden, sharp corners]}{\end{tcolorbox}}\fi

Incidence geometry is one of the most important branches of mathematics.
It deals with the properties and relationships between points and lines.
We'll will take a closer look at the different aspects of incidence
geometry and see what readers can expect to learn from it.

First, we study the three Incidence Axioms as laid out by David Hilbert
in his infamous work \emph{Foundations of Geometry}. We prove several
elementary statements using these incidence axioms.~ Then we discuss the
possibility of adding an additional axiom to the system: a parallel
postulate.~ To motivate the discussion, we work through a few finite
models of incidence axioms, as we demonstrate the independence of the
proposed parallel postulates. Next, we define the Affine Plane:
basically the Incidence Axioms plus a parallel postulate. Then we
explore the basics of finite affine planes and end with the major open
problem in finite geometry. ~

\hypertarget{what-do-we-need-for-the-foundations-of-geometry}{%
\subsection{What do we need for the foundations of
geometry?}\label{what-do-we-need-for-the-foundations-of-geometry}}

Incidence geometry is based on the relationships between points and
lines. The incidence axioms are a set of principles that govern these
objects and their relationship.

The first incidence axiom states that any two points can be connected by
a unique line. The second incidence axiom states that any two lines
intersect at a unique point. The third incidence axiom states that there
exist three non-collinear points. These three incidence axioms are the
foundation for all of the incidence geometry.

One might think that all we need for geometry is a straightedge and
compass, but incidence geometry shows us that this isn't the case. In
incidence geometry, we don't use any measurements; instead, we only look
at how objects are related to one another. This means that we can
explore geometry without worrying about numbers or angles. So, what do
we need for the foundations of incidence geometry? Simply a few basic
concepts: \textbf{points}, \textbf{lines}, and an \textbf{incidence
relation}. With these concepts in hand, we can begin to investigate the
fascinating world of geometry.

\hypertarget{what-do-we-need-to-do-geometry}{%
\subsection{What do we need to do
geometry?}\label{what-do-we-need-to-do-geometry}}

The short answer is \textbf{The Axiomatic Method}. But let's give a few
details but what we mean by this.

\hypertarget{the-axiomatic-method}{%
\subsubsection{The Axiomatic Method}\label{the-axiomatic-method}}

An axiomatic system must contain a set of technical terms that are
deliberately chosen as \textbf{undefined terms} and are subject to the
interpretation of the reader. All other technical terms of the system
are \textbf{defined terms} whose definition rely upon the undefined
terms. The \textbf{axioms} of the system are a set of statements dealing
with the defined and undefined terms and will remain unproven. The word
postulate is a synonym and is often used in place of the word axiom.

All other statements must be logical consequences of the axioms. These
derived statements are called the \textbf{theorems} of the axiomatic
system.

Thus, in order to follow the axiomatic method, and do any type of
mathematics, we require the following:

\begin{itemize}
\tightlist
\item
  Mutual understanding of the meaning of the words and symbols used in
  the discourse.
\item
  Acceptance of certain statements called axioms or postulates without
  further justification.
\item
  Agreement on how and when one statement ``follows logically'' from
  another, i.e., agreement on certain rules of logic.
\end{itemize}

Here is more information about what I mean by this:

\begin{enumerate}
\def\labelenumi{\arabic{enumi}.}
\tightlist
\item
  No unstated assumptions may be used in a proof.
\item
  The following are the six types of justifications allowed for
  statements in proofs:
\end{enumerate}

\begin{itemize}
\tightlist
\item
  By hypothesis
\item
  By axiom
\item
  By theorem
\item
  By definition
\item
  By step
\item
  By rule of logic
\end{itemize}

\begin{enumerate}
\def\labelenumi{\arabic{enumi}.}
\setcounter{enumi}{2}
\tightlist
\item
  To prove a statement \(P\implies Q\) assume the negation of statement
  \(Q\) (RAA hypothesis) and deduce an absurd statement, using the
  hypothesis \(P\) if needed in your deduction.
\item
  The statement \(\neg (\neg P)\) means the same as \(P\).
\item
  The statement \(\neg (P\implies Q)\) means the same as
  \(P\land \neg Q\).
\item
  The statement \(\neg(P\land Q)\) means the same as
  \(\neg P \lor \neg Q\).
\item
  The statement \(\neg(\forall x P(x))\) means the same as
  \(\exists x \neg P(x)\).
\item
  The statement \(\neg(\exists x P(x))\) means the same as
  \(\forall x \neg P(x)\).
\item
  If \(P\implies Q\) and \(P\) are steps in a proof, then \(Q\) is a
  justifiable step.
\item
  \((P\implies Q) \land (Q\implies R) \implies (P\implies R)\)
\item
  \((P\land Q)\implies P\) and \((P\land Q)\implies Q\)
\item
  \((\neg Q \implies \neg P)\Longleftrightarrow (P\implies Q)\)
\item
  For every statement \(P\), \(P\lor \neg P\) is a valid step in a
  proof.
\item
  Suppose the disjunction of statements \(P_1\) or \(P_2\) or \(\ldots\)
  or \(P_n\) is already a valid step in a proof. Suppose that proofs of
  \(Q\) are carried out from each of the case assumptions \(P_1\),
  \(P_2\), \(\ldots\), \(P_n.\) Then \(Q\) can be concluded as a valid
  step in the proof (proof by cases).
\item
  \(\forall X \, (X=X)\)
\item
  \(\forall X \, \forall Y \, (X=Y \Longleftrightarrow Y=X)\)
\item
  \(\forall X \, \forall Y \, \forall Z \, ( (X=Y) \land (Y=Z) ) \implies (X=Z)\)
\item
  If \(X=Y\) and \(P(X)\) is a statement about \(X\), then
  \(P(X)=P(Y)\).
\end{enumerate}

\hypertarget{properties-of-an-axiomatic-system}{%
\subsection{Properties of an Axiomatic
System}\label{properties-of-an-axiomatic-system}}

Axiomatic systems have properties the most essential of which is
\textbf{consistency}. An axiomatic system is consistent provided any
theorem which can be proven using the system can never logically
contradict any of the axioms or previously proved theorems. An
individual axiom is \textbf{independent} if it cannot be proven by use
of the other axioms. An axiomatic system is independent provided each of
its axioms is independent. An axiom set must also be \textbf{complete}.
We must have enough axioms that every theorem that can possibly be
stated using the terms and axioms of the system can be proven either
true or false.

\hypertarget{incidence-planes}{%
\subsection{Incidence Planes}\label{incidence-planes}}

There are many expressions used to describe \textbf{incidence} (for
example, \textbf{a line passes through a point}, \textbf{a point lies on
a line}, or \textbf{two lines intersect at a point}. However, the term
\textbf{incidence} is preferred because it does not have the additional
connotations that these other terms have.

\hypertarget{axioms-and-definitions}{%
\subsubsection{Axioms and Definitions}\label{axioms-and-definitions}}

Let \textbf{point}, \textbf{line}, and \textbf{incidence} be undefined
terms. Collectively the following three axioms are called the
\textbf{Incidence Axioms}.

\begin{definition}[]\protect\hypertarget{def-incidence-plane}{}\label{def-incidence-plane}

An \textbf{incidence plane} consists of a set of points, and a set of
lines, satisfying the axioms:

\end{definition}

\begin{itemize}
\tightlist
\item
  (A1) For every two distinct points \(A\) and \(B\) there exists a
  unique line \(l\) incident with \(A\) and \(B\), and is denoted by
  \(l(A,B)\), or sometimes by \(\overleftrightarrow{AB}\).
\item
  (A2) For every line \(l\) there exist at least two distinct points
  incident with \(l\).
\item
  (A3) There exist three distinct points with the property that no line
  is incident with all three of them.
\end{itemize}

\begin{definition}[]\protect\hypertarget{def-collinear-concurrent}{}\label{def-collinear-concurrent}

Three or more points (lines) are called \textbf{collinear}
(\textbf{concurrent}) if there exists a line (point) incident with all
of them.

\end{definition}

We use the terms \textbf{noncollinear} and \textbf{nonconcurrent} to
mean not collinear and not concurrent, respectively. Lines \(l\) and
\(m\) are called \textbf{equal}, denoted by \(l=m\), if every point
incident with \(l\) is also incident with \(m\), and conversely.

\begin{definition}[]\protect\hypertarget{def-parallel}{}\label{def-parallel}

Lines \(l\) and \(m\) are called \textbf{parallel}, denoted by
\(l\parallel m\), whenever \(l\neq m\) and they have no points in
common.

\end{definition}

Here are some examples using logic rules. In the following, we denote
\(P \vert l\) to mean point \(P\) is incidence with line \(l\).

(A1) For every point \(P\) and for every point \(Q\) not equal to \(P\),
there exists a unique line \(l\) incident with \(P\) and \(Q.\)

\[\forall P\, \forall Q\, ((P\neq Q)\implies \exists ! \, l (P|l \land Q| l))\]

The negation is:

\[\exists P \, \exists Q\, \left[ P\neq Q \land ( (\exists l \, \exists m \, (l\neq m \land P|l \land Q|l)) \lor (\neg \, \exists l \, (P|l \land Q|l) ) ) \right]\]

(A2) For every line \(l\), there exists at least two distinct points
incident with \(l\).

\[\forall l\, \exists P \, \exists Q \, (P\neq Q \land (P|l \land Q|l))\]

The negation is:

\[\forall l\, \exists P \, \exists Q \, (P=Q \lor ( \neg P|l \lor \neg Q|l))\]

(A3) There exist three distinct points with the property that no line is
incident with all three of them.

\[\exists P\, \exists Q\, \exists R\, (P\neq Q \land P\neq R \land Q\neq R) \land \neg \, \exists l\, (P|l \land Q|l \land R|l)\]

The negation is:

\[\forall P\, \forall Q\, \forall R\, (P=Q \lor P= R \lor Q = R) \lor  \exists l\, (P|l \land Q|l \land R|l)\]

\begin{theorem}[]\protect\hypertarget{thm-one-col}{}\label{thm-one-col}

If two distinct lines are not parallel, then they have a unique point in
common.

\end{theorem}

\begin{longtable}[]{@{}
  >{\raggedright\arraybackslash}p{(\columnwidth - 4\tabcolsep) * \real{0.1538}}
  >{\raggedright\arraybackslash}p{(\columnwidth - 4\tabcolsep) * \real{0.3846}}
  >{\raggedleft\arraybackslash}p{(\columnwidth - 4\tabcolsep) * \real{0.4615}}@{}}
\toprule()
\begin{minipage}[b]{\linewidth}\raggedright
Step
\end{minipage} & \begin{minipage}[b]{\linewidth}\raggedright
Statement
\end{minipage} & \begin{minipage}[b]{\linewidth}\raggedleft
Justification
\end{minipage} \\
\midrule()
\endhead
1 & \(l \neq m\) & hypothesis \\
2 & \(l\) and \(m\) are not \(\parallel\) & hypothesis \\
3 & there exists a point \(P\) that lies on \(l\) and \(m\) & negation
of \(parallel\) \\
4 & there exists a point \(Q\) incident with \(l\) and \(m\) with
\(Q\neq P\) & RAA hypothesis \\
5 & \(l = m\) & (A1) \\
6 & \(\Leftrightarrow\) & Steps (1) and (5) \\
7 & \(P\) is unique & RAA conclusion \\
\bottomrule()
\end{longtable}

\begin{theorem}[]\protect\hypertarget{thm-two-col}{}\label{thm-two-col}

There exist three distinct lines that are nonconcurrent.

\end{theorem}

\begin{longtable}[]{@{}
  >{\raggedright\arraybackslash}p{(\columnwidth - 4\tabcolsep) * \real{0.1538}}
  >{\raggedright\arraybackslash}p{(\columnwidth - 4\tabcolsep) * \real{0.3846}}
  >{\raggedleft\arraybackslash}p{(\columnwidth - 4\tabcolsep) * \real{0.4615}}@{}}
\toprule()
\begin{minipage}[b]{\linewidth}\raggedright
Step
\end{minipage} & \begin{minipage}[b]{\linewidth}\raggedright
Statement
\end{minipage} & \begin{minipage}[b]{\linewidth}\raggedleft
Justification
\end{minipage} \\
\midrule()
\endhead
1 & \(A,\) \(B,\) and \(C\) are distinct noncollinear points & Axiom
3 \\
2 & there exists a line \(l(AB)\) incident with \(A\) and \(B\) & Axiom
1 \\
3 & there exists a line \(l(BC)\) incident with \(B\) and \(C\) & Axiom
1 \\
4 & there exists a line \(l(AC)\) incident with \(A\) and \(C\) & Axiom
1 \\
5 & \(l(AB)=l(BC)\) & RAA Hypothesis \\
6 & \(A,\) \(B,\) and \(C\) are collinear & Steps 2, 3, 5 \\
7 & \(\rightarrow \leftarrow\) & Steps 1, 6 \\
8 & \(l(AB)\neq l(BC)\) & RAA Conclusion \\
9 & \(l(AB)=l(AC)\) & RAA Hypothesis \\
10 & \(A,\) \(B,\) and \(C\) are collinear & Steps 2, 4, 9 \\
11 & \(\rightarrow \leftarrow\) & Steps 1, 10 \\
12 & \(l(AB)\neq l(AC)\) & RAA Conclusion \\
13 & \(l(AC)=l(BC)\) & RAA Hypothesis \\
14 & \(A,\) \(B,\) and \(C\) are collinear & Steps 3, 4, 13 \\
15 & \(\rightarrow \leftarrow\) & Steps 1, 14 \\
16 & \(l(AC)\neq l(BC)\) & RAA Conclusion \\
17 & lines \(l(AB),\) \(l(BC),\) \(l(AC)\) are three distinct lines &
Steps 8, 12, 16 \\
18 & there exists a point \(X\) incident with all three lines \(l(AB),\)
\(l(BC),\) \(l(AC)\) & RAA Hypothesis \\
19 & one and only one must hold: \(X=A\) or \(X\neq A\) & Law of
Excluded Middle \\
20 & \(X=A\) & Case 1 \\
21 & \(A\) is incident with all three lines \(l(AB),\) \(l(BC),\)
\(l(AC)\) & Steps 18, 20 \\
22 & \(A,\) \(B,\) \(C\) are collinear & Definition of Collinear \\
23 & \(\rightarrow \leftarrow\) & Steps 1, 22 \\
24 & \(X\neq A\) & Case 2 \\
25 & lines \(l(AB)\) and \(l(AC)\) are not parallel & Definition of
Parallel Lines \\
26 & \(X=A\) & Theorem 1 \\
27 & \(\rightarrow \leftarrow\) & Steps 24, 26 \\
28 & there does not exist a point \(X\) incident with all three lines
\(l(AB),\) \(l(BC),\) \(l(AC)\) & RAA Conclusion \\
29 & lines \(l(AB),\) \(l(BC),\) \(l(AC)\) are nonconcurrent &
Definition of nonconcurrent \\
\bottomrule()
\end{longtable}

\begin{theorem}[]\protect\hypertarget{thm-five-col}{}\label{thm-five-col}

For every point \(A,\) there is at least one line not passing through
\(A.\)

\end{theorem}

\begin{longtable}[]{@{}
  >{\raggedright\arraybackslash}p{(\columnwidth - 4\tabcolsep) * \real{0.1538}}
  >{\raggedright\arraybackslash}p{(\columnwidth - 4\tabcolsep) * \real{0.3846}}
  >{\raggedleft\arraybackslash}p{(\columnwidth - 4\tabcolsep) * \real{0.4615}}@{}}
\toprule()
\begin{minipage}[b]{\linewidth}\raggedright
Step
\end{minipage} & \begin{minipage}[b]{\linewidth}\raggedright
Statement
\end{minipage} & \begin{minipage}[b]{\linewidth}\raggedleft
Justification
\end{minipage} \\
\midrule()
\endhead
1 & \(A\) is a point & Hypothesis \\
2 & every line passes through \(A\) & RAA Hypothesis \\
3 & there exists 3 noncollinear points \(E,\) \(D,\) \(F\) & Axiom 3 \\
4 & there exists a line \(l(ED)\) incident with \(E\) and \(D\) & Axiom
1 \\
5 & there exists a line \(l(DF)\) incident with \(D\) and \(F\) & Axiom
1 \\
6 & one and only one must hold: \(A=D\) or \(A\neq D\) & Law of Excluded
Middle \\
7 & \(A\neq D\) & Case 1 \\
8 & \(A\) and \(D\) are incident with \(l(ED)\) and \(l(DF)\) & Steps 2,
4, 5 \\
9 & \(l(ED)\)= \(l(DF)\) & Axiom 1 \\
10 & \(F\) is incident with \(l(ED)\) & Definition of Equal Lines \\
11 & \(E,\) \(D,\) and \(F\) are collinear & Definition of collinear \\
12 & \(\rightarrow \leftarrow\) & Steps 3 and 11 \\
13 & \(A=D\) & Case 2 \\
14 & every line is incident with \(D\) & Steps 2, 13 \\
15 & there exists a line \(l(EF)\) incident with \(E\) and \(F\) & Axiom
1 \\
16 & \(D\) is incident with \(l(EF)\) & Axiom 1 \\
17 & \(E,\) \(D,\) and \(F\) are collinear & Definition of Collinear \\
18 & \(\rightarrow \leftarrow\) & Steps 3, 17 \\
19 & there exists a line not incident with \(A\) & RAA conclusion \\
\bottomrule()
\end{longtable}

\hypertarget{basic-theorems}{%
\subsection{Basic Theorems}\label{basic-theorems}}

Some of the theorems below have a column proof in the appendix. For the
theorems that do not have a column proof the reader is urge to provide
one.

\begin{theorem}[]\protect\hypertarget{thm-one}{}\label{thm-one}

If two distinct lines are not parallel, then they have a unique point in
common.

\end{theorem}

\begin{proof}

Suppose \(l\) and \(m\) are distinct lines. Since \(l\) and \(m\) are
not parallel, there exists a point incidence with both of them, say
\(P.\) Let \(Q\) be a point incident with both of them. Assume
\(P\neq Q.\) By axiom (A1), it follows that \(l=m,\) contrary to
hypothesis. Hence, \(P=Q\) and so \(P\) is unique. \(\square\)

\end{proof}

\begin{theorem}[]\protect\hypertarget{thm-two}{}\label{thm-two}

There exist three distinct lines that are nonconcurrent.

\end{theorem}

\begin{proof}

By Axiom 3, there exists three distinct noncolinear points, say
\(A, B, C.\) By Axiom 1, we have lines \(l(AB),\) \(l(BC),\) and
\(l(AC).\) If any one of these lines is equal to another, then all three
points are collinear. Hence, these three lines are distinct. Assume for
a contradiction that there exists a point \(X\) incident with all three
lines. If \(X,\) then \(A, B, C\) are collinear contrary to hypothesis.
If \(X\neq A,\) then distinct lines \(l(AB)\) and \(l(AC)\) do not have
a unique point in common, contrary to Theorem 1. Hence, \(X\) does not
exists. Therefore, the lines \(l(AB),\) \(l(BC),\) and \(l(AC)\) are
nonconcurrent. \(\square\)

\end{proof}

\begin{theorem}[]\protect\hypertarget{thm-three}{}\label{thm-three}

Every point is incident with at least one line.

\end{theorem}

\begin{proof}

Suppose that \(P\) is a point that no line is incidence with. By (A3),
there exists three distinct noncolinear points, say \(A, B, C.\) Without
loss of generality, say \(P\neq A,\) and so by (A1) we see that there
exists a line incident with \(P,\) contrary to hypothesis. \(\square\)

\end{proof}

\begin{theorem}[]\protect\hypertarget{thm-four}{}\label{thm-four}

Every line has at least one point not incident with it.

\end{theorem}

\begin{proof}

Suppose there exists a line, say \(l,\) that is incident with every
point. By (A3), there exists three distinct noncolinear points, say
\(A, B, C,\) contrary to hypothesis. \(\square\)

\end{proof}

\begin{theorem}[]\protect\hypertarget{thm-five}{}\label{thm-five}

Every point has at least one line not incident with it.

\end{theorem}

\begin{proof}

Assume for a contradiction that \(A\) is a point that every line is
incident with. By (A3), there exists three distinct noncolinear points,
say \(E, D, F.\) By (A1), we then see that \(A\) is incident with the
three lines \(l(ED),\) \(l(DF),\) and \(l(EF).\) If \(A=D,\) then
\(E, D, F\) are incident with line \(l(EF).\) If \(A\neq D,\) then
\(l(ED)=l(DF)=l(AD)\) by (A1). Hence, \(E, D, F\) are incident with line
\(l(AD).\) This contradiction shows that there exists a line that is not
incident with \(A.\) \(\square\)

\end{proof}

\begin{theorem}[]\protect\hypertarget{thm-six}{}\label{thm-six}

Every point is incident with at least two distinct lines.

\end{theorem}

\begin{proof}

Let \(P\) be a point. By Axiom 3, there exists three distinct
noncolinear points, say \(A, B, C.\) Since \(A, B, C\) are noncollinear,
it follows that \(l(AB),\) \(l(AC),\) and \(l(BC)\) are three distinct
lines. Suppose that \(P\) is one of these points, \(A,\) \(B,\) or
\(C.\) Without loss of generality, say \(A.\) By (A1), it follows that
\(l(AB)\) and \(l(AC)\) are two distinct lines that are incident with
\(P.\) Suppose that \(P,\) \(A, B, C\) are four distinct points. By
(A1), there exists lines \(l(AP)\) incident with \(A\) and \(P,\)
\(l(BP)\) incident with \(B\) and \(P,\) and there exists \(l(CP)\)
incident with \(C\) and \(P.\) If \(l(AP)=l(BP)=l(CP),\) then
\(A, B, C\) are collinear. Hence, there are at least two lines passing
through \(P.\) \(\square\)

\end{proof}

\begin{theorem}[]\protect\hypertarget{thm-seven}{}\label{thm-seven}

If \(C\) is incident with \(l(AB)\) and distinct from \(A\) and \(B,\)
then \(l(CA)=l(BC)=l(AB).\)

\end{theorem}

\begin{proof}

By (A1), it follows \(l(CA)=l(AB)\) since \(C\) and \(A\) are distinct
and incident with \(l(AB).\) By (A1), it follows \(l(BC)=l(AB)\) since
\(C\) and \(B\) are distinct and incident with \(l(AB).\) Hence
\(l(CA)=l(BC)=l(AB).\) \(\square\)

\end{proof}

\begin{theorem}[]\protect\hypertarget{thm-eight}{}\label{thm-eight}

If \(l(AB)=l(AC)\) and \(B\) and \(C\) are distinct, then
\(l(AB)=l(BC).\)

\end{theorem}

\begin{proof}

By (A1), it follows \(l(AB)=l(BC)\) since \(C\) and \(B\) are distinct
and incident with \(l(AB).\) \(\square\)

\end{proof}

\begin{theorem}[]\protect\hypertarget{thm-nine}{}\label{thm-nine}

If \(l\) is any line, then there exists lines \(m\) and \(n\) such that
\(l,\) \(m,\) and \(n\) are distinct and both \(m\) and \(n\) have a
point in common with \(l.\)

\end{theorem}

\begin{proof}

Suppose that \(l\) is a line. By Theorem 4 , there exists a point \(P\)
not incident with \(l.\) By (A2), there exists points \(A\) and \(B\)
both incident with \(l.\) Then \(A\neq P\) and \(B\neq P,\) since \(A\)
and \(B\) are incident with \(l\) and \(P\) is not incident with \(l.\)
By (A1), there exists lines \(m:=l(AP)\) and \(n:=l(BP)\) and both of
these lines have a point in common with \(l.\) Clearly, \(l,\) \(m,\)
and \(n\) are distinct lines.\(\square\)

\end{proof}

\begin{theorem}[]\protect\hypertarget{thm-ten}{}\label{thm-ten}

If \(A\) is any point, then there exist points \(B\) and \(C\) such that
\(A,\) \(B,\) and \(C\) are noncollinear.

\end{theorem}

\begin{proof}

Let \(A\) be a point. By Theorem 5 , there exists a line not incident
with \(A,\) call it \(l.\) By (A2), there exists two distinct points
\(B\) and \(C\) incident with line \(l.\) Assume for a contradiction
that \(A,\) \(B,\) and \(C\) are collinear to line \(m.\) By (A1),
\(m=l\) and so \(A\) is incident with \(l.\) This contradiction show
that \(A,\) \(B,\) and \(C\) can not be collinear. \(\square\)

\end{proof}

\begin{theorem}[]\protect\hypertarget{thm-eleven}{}\label{thm-eleven}

If \(A\) and \(B\) are two distinct points, then there exists a point
\(C\) such that \(A,\) \(B,\) and \(C\) are noncollinear.

\end{theorem}

\begin{proof}

Let \(A\) and \(B\) be two distinct points. By (A1), there exists line
\(l=l(AB),\) and by Theorem 4 there exists a points \(C\) not incident
with \(l.\) Assume for a contradiction that \(A,\) \(B,\) and \(C\) are
collinear to line \(m.\) By (A1), \(m=l\) and so \(C\) is incident with
\(l,\) contrary to hypothesis. This contradiction show that \(A,\)
\(B,\) and \(C\) can not be collinear. \(\square\)

\end{proof}

\hypertarget{parallelism}{%
\subsection{Parallelism}\label{parallelism}}

Consider the following statement:

\begin{quote}
In a plane, given a line and a point not on it, at most one line
parallel to the given line can be drawn through the point.
\end{quote}

This statement is historical called \textbf{Playfair's axiom} (named
after the Scottish mathematician John Playfair) and is an axiom that can
be used instead of the fifth postulate of Euclid (the Parallel
postulate) when working with Euclidean geometry.

\begin{quote}
In David Hilbert's book, \textbf{Foundations of Geometry} (1899), he
provided a new set of axioms for Euclidean geometry using Playfair's
axiom instead of Euclid's version.
\end{quote}

Here are three additional statements we will consider. These are not new
axioms for incidence geometry. Rather they are additional statements
that may or may not be satisfied by a particular model for incidence
geometry.

\begin{itemize}
\tightlist
\item
  (\textbf{Elliptic Parallel Property}) For any line \(l\) and any point
  \(P\) not incident with \(l,\) there exist no lines incident with
  \(P\) that are parallel to \(l.\)
\item
  (\textbf{Euclidean Parallel Property}) For any line \(l\) and any
  point \(P\) not incident with \(l,\) there exists exactly one line
  incident with \(P\) that is parallel to \(l.\)
\item
  (\textbf{Hyperbolic Parallel Property}) For any line \(l\) and any
  point \(P\) not incident with \(l,\) there exists more than one line
  incident with \(P\) parallel to \(l.\)
\end{itemize}

Can any of these three statements be proven from the Incidence Axioms?
If not, how would one show that and which one should we take as an
additional axiom?

Before we answer such questions let's consider three basic properties of
parallelism, namely the reflexive, symmetric, and transitive properties.

\begin{definition}[]\protect\hypertarget{def-parallel}{}\label{def-parallel}

Let \(l\sim m\) mean ``line \(l\) is parallel to line \(m\) or equal to
\(m\)''.

\end{definition}

\begin{proposition}[]\protect\hypertarget{prp-one}{}\label{prp-one}

Parallelism is reflexive and symmetric. More precisely, the following
hold:

\begin{itemize}
\tightlist
\item
  (\textbf{reflexive}) if \(l\) is a line, then \(l \parallel l\); and
\item
  (\textbf{symmetric}) if \(l\) and \(m\) are lines, then
  \(l \parallel m\) implies \(m \parallel l.\)
\end{itemize}

\end{proposition}

\begin{proposition}[Parallelism is
Transitive]\protect\hypertarget{prp-two}{}\label{prp-two}

If the Euclidean Parallel Property holds, then \(\sim\) is an
equivalence relation.

\end{proposition}

\begin{proof}

Assume that the Euclidean Parallel Property holds. Suppose \(l\sim m\)
and \(m\sim n.\) Also suppose that \(l\sim n\) does not hold. Since
\(\neg(l\sim n),\) we have \(l\neq n\) and by Theorem 1, there exists a
point \(P\) incident with lines \(l\) and \(n.\) Notice that \(l=m\) and
\(m=n\) can not happen since \(l\neq n.\) Suppose that \(l=m\) and
\(m \parallel n.\) Then \(P\) lies on \(n, l,\) and \(m\) which can not
occur because \(m \parallel n.\) Similarly, \(m=n\) and
\(l \parallel m\) can not occur. The remaining case, of
\(l \parallel m\) and \(m \parallel n\) can not occur either since the
Euclidean Parallel Property holds. All cases considered, it follows that
if \(l\sim m\) and \(m\sim n,\) then \(l\sim n\) must hold. Therefore,
\(\sim\) is reflexive, symmetric, and transitive as needed.\(\square\)

\end{proof}

We will now see that none of these parallel properties can be proven
from the Incidence Axioms. Moreover, you might have noticed that, at
most one of them can hold at any given time.

\hypertarget{models-of-finite-incidence-planes}{%
\section{Models of Finite Incidence
Planes}\label{models-of-finite-incidence-planes}}

Any collection of points and lines which satisfies the Incident Axioms
(A1), (A2), (A3) is called an \textbf{incidence geometry}. In this
section we give several concrete examples (\textbf{models}) of incidence
geometry.

\hypertarget{three-point-geometry}{%
\subsection{Three-Point Geometry}\label{three-point-geometry}}

In Three-Point Geometry, we interpret a point to be one of the symbols
\(A,\) \(B,\) or \(C\) and a line to consist of exactly two points.
Explicitly,

\begin{longtable}[]{@{}lll@{}}
\toprule()
\(l_1\) & \(l_2\) & \(l_3\) \\
\midrule()
\endhead
\(A\) & \(A\) & \(B\) \\
\(B\) & \(C\) & \(C\) \\
\bottomrule()
\end{longtable}

Notice the following hold: - Each point is incident with two lines. -
There are no parallel lines, and so the Elliptic Parallel Property
holds. \(\square\)

Any collection of points and lines which satisfies the Incident Axioms
is called an \textbf{incidence geometry}.

Is incidence geometry consistent? To answer this question, let's
hypothesize that incidence geometry is inconsistent. Then the statement:

\begin{quote}
if points \(A\) and \(B\) are distinct, then \(A\) and \(B\) are the
same point
\end{quote}

could be proven. If such a proof existed in incidence geometry then it
would hold in any model of it. In particular, a proof of such a
statement would hold in Three-Point Geometry. However, this is
impossible since there are exactly three distinct points. Therefore, we
can conclude, that incidence geometry is consistent. This is an example
of how a model can serve as evidence of the consistency of an axiomatic
system.

\hypertarget{four-point-geometry}{%
\subsection{Four-Point Geometry}\label{four-point-geometry}}

In Four-Point Geometry, we interpret the symbols \(A,\) \(B,\) \(C,\)
and \(D\) to be points and there are precisely six lines each of which
consists of only two distinct points as follows.

\begin{longtable}[]{@{}llllll@{}}
\toprule()
\(l_1\) & \(l_2\) & \(l_3\) & \(l_4\) & \(l_5\) & \(l_6\) \\
\midrule()
\endhead
\(A\) & \(A\) & \(A\) & \(B\) & \(B\) & \(C\) \\
\(B\) & \(C\) & \(D\) & \(C\) & \(D\) & \(D\) \\
\bottomrule()
\end{longtable}

Notice the following hold: - Each point is incident with three lines. -
Each distinct line has exactly one line parallel to it and so the
Euclidean Parallel Property holds. \(\square\)

\hypertarget{five-point-geometry}{%
\subsection{Five-Point Geometry}\label{five-point-geometry}}

In Five-Point Geometry, we interpret the symbols \(A,\) \(B,\) \(C,\)
\(D,\) and \(E\) to be points and there are precisely ten lines each of
which consists of only two distinct points as follows.

\begin{longtable}[]{@{}
  >{\raggedright\arraybackslash}p{(\columnwidth - 18\tabcolsep) * \real{0.1000}}
  >{\raggedright\arraybackslash}p{(\columnwidth - 18\tabcolsep) * \real{0.1000}}
  >{\raggedright\arraybackslash}p{(\columnwidth - 18\tabcolsep) * \real{0.1000}}
  >{\raggedright\arraybackslash}p{(\columnwidth - 18\tabcolsep) * \real{0.1000}}
  >{\raggedright\arraybackslash}p{(\columnwidth - 18\tabcolsep) * \real{0.1000}}
  >{\raggedright\arraybackslash}p{(\columnwidth - 18\tabcolsep) * \real{0.1000}}
  >{\raggedright\arraybackslash}p{(\columnwidth - 18\tabcolsep) * \real{0.1000}}
  >{\raggedright\arraybackslash}p{(\columnwidth - 18\tabcolsep) * \real{0.1000}}
  >{\raggedright\arraybackslash}p{(\columnwidth - 18\tabcolsep) * \real{0.1000}}
  >{\raggedright\arraybackslash}p{(\columnwidth - 18\tabcolsep) * \real{0.1000}}@{}}
\toprule()
\begin{minipage}[b]{\linewidth}\raggedright
\(l_1\)
\end{minipage} & \begin{minipage}[b]{\linewidth}\raggedright
\(l_2\)
\end{minipage} & \begin{minipage}[b]{\linewidth}\raggedright
\(l_3\)
\end{minipage} & \begin{minipage}[b]{\linewidth}\raggedright
\(l_4\)
\end{minipage} & \begin{minipage}[b]{\linewidth}\raggedright
\(l_5\)
\end{minipage} & \begin{minipage}[b]{\linewidth}\raggedright
\(l_6\)
\end{minipage} & \begin{minipage}[b]{\linewidth}\raggedright
\(l_7\)
\end{minipage} & \begin{minipage}[b]{\linewidth}\raggedright
\(l_8\)
\end{minipage} & \begin{minipage}[b]{\linewidth}\raggedright
\(l_9\)
\end{minipage} & \begin{minipage}[b]{\linewidth}\raggedright
\(l_{10}\)
\end{minipage} \\
\midrule()
\endhead
\(A\) & \(B\) & \(C\) & \(D\) & \(A\) & \(B\) & \(B\) & \(A\) & \(C\) &
\(A\) \\
\(B\) & \(C\) & \(D\) & \(E\) & \(C\) & \(D\) & \(E\) & \(D\) & \(E\) &
\(E\) \\
\bottomrule()
\end{longtable}

Notice the following hold: - Each line is incident with exactly two
points. - Each point is incident with four lines. - Each distinct line
has exactly three lines parallel to it. - The Hyperbolic Parallel
Property holds. \(\square\)

\textbf{Remark}. Notice that, in Three-Point Geometry the Elliptic
Parallel Property is true, in Four-Point Geometry the Euclidean Parallel
Property is true, and in Five-Point Geometry the Hyperbolic Parallel
Property is true. Thus Three-Point, Four Point, and Five-Point
geometries reveal that the three Parallel Properties are all independent
from the Incidence Axioms.

\hypertarget{fanos-geometry}{%
\subsection{Fano's Geometry}\label{fanos-geometry}}

The Italian mathematician Gino Fano developed the first finite geometry.
He produced a finite three dimensional space with 15 points, 35 lines
and 15 planes, in which each line had only three points on it. He did
this while working on proving the independence of his set of axioms for
his \(n\)-dimensional geometry. The planes in this space (now known as
Fano planes) consist of seven points and seven lines.

\begin{example}[]\protect\hypertarget{exm-ano-geometry}{}\label{exm-ano-geometry}

The following interpretation, called Fano's Geometry, is a model for
incidence geometry. Each point is one of the seven symbols \(A,\) \(B,\)
\(C,\) \(D,\) \(E,\) \(F\) or \(G\) and a point \(P\) is incident with a
line \(l\) means \(l\) consists only of the point \(P\) and two other
distinct points. There are exactly seven lines where each line consists
of exactly three points and shown.

\begin{longtable}[]{@{}lllllll@{}}
\toprule()
\(l_1\) & \(l_2\) & \(l_3\) & \(l_4\) & \(l_5\) & \(l_6\) & \(l_7\) \\
\midrule()
\endhead
A & A & A & B & C & C & E \\
B & G & E & G & G & F & B \\
C & F & D & D & E & D & F \\
\bottomrule()
\end{longtable}

The following statements are true in Fano's geometry: - the Elliptic
Parallel Property holds, - each point is incident with exactly three
lines, and - for every two distinct points there are exactly two lines
that are not incident with these points. \(\square\)

\end{example}

\hypertarget{youngs-geometry}{%
\subsection{Young's Geometry}\label{youngs-geometry}}

The following interpretation, called Young's Geometry, is a model for
incidence geometry. Each point is one of the nine symbols \(A,\) \(B,\)
\(C,\) \(D,\) \(E,\) \(F,\) \(G,\) \(H,\) \(I\) and a point \(P\) is
incident with a line \(l\) means \(l\) consists only of the point \(P\)
and two other distinct points. There are exactly twelve lines where each
line consists of exactly three points and shown.

\begin{longtable}[]{@{}
  >{\raggedright\arraybackslash}p{(\columnwidth - 20\tabcolsep) * \real{0.0909}}
  >{\raggedright\arraybackslash}p{(\columnwidth - 20\tabcolsep) * \real{0.0909}}
  >{\raggedright\arraybackslash}p{(\columnwidth - 20\tabcolsep) * \real{0.0909}}
  >{\raggedright\arraybackslash}p{(\columnwidth - 20\tabcolsep) * \real{0.0909}}
  >{\raggedright\arraybackslash}p{(\columnwidth - 20\tabcolsep) * \real{0.0909}}
  >{\raggedright\arraybackslash}p{(\columnwidth - 20\tabcolsep) * \real{0.0909}}
  >{\raggedright\arraybackslash}p{(\columnwidth - 20\tabcolsep) * \real{0.0909}}
  >{\raggedright\arraybackslash}p{(\columnwidth - 20\tabcolsep) * \real{0.0909}}
  >{\raggedright\arraybackslash}p{(\columnwidth - 20\tabcolsep) * \real{0.0909}}
  >{\raggedright\arraybackslash}p{(\columnwidth - 20\tabcolsep) * \real{0.0909}}
  >{\raggedright\arraybackslash}p{(\columnwidth - 20\tabcolsep) * \real{0.0909}}@{}}
\toprule()
\begin{minipage}[b]{\linewidth}\raggedright
\(l_1\)
\end{minipage} & \begin{minipage}[b]{\linewidth}\raggedright
\(l_2\)
\end{minipage} & \begin{minipage}[b]{\linewidth}\raggedright
\(l_3\)
\end{minipage} & \begin{minipage}[b]{\linewidth}\raggedright
\(l_4\)
\end{minipage} & \begin{minipage}[b]{\linewidth}\raggedright
\(l_5\)
\end{minipage} & \begin{minipage}[b]{\linewidth}\raggedright
\(l_6\)
\end{minipage} & \begin{minipage}[b]{\linewidth}\raggedright
\(l_7\)
\end{minipage} & \begin{minipage}[b]{\linewidth}\raggedright
\(l_8\)
\end{minipage} & \begin{minipage}[b]{\linewidth}\raggedright
\(l_9\)
\end{minipage} & \begin{minipage}[b]{\linewidth}\raggedright
\(l_{10}\)
\end{minipage} & \begin{minipage}[b]{\linewidth}\raggedright
\(l_{11}\)
\end{minipage} \\
\midrule()
\endhead
\(A\) & \(A\) & \(A\) & \(A\) & \(B\) & \(B\) & \(B\) & \(C\) & \(C\) &
\(C\) & \(D\) \\
\(B\) & \(D\) & \(E\) & \(F\) & \(D\) & \(E\) & \(F\) & \(D\) & \(E\) &
\(F\) & \(E\) \\
\(C\) & \(G\) & \(H\) & \(I\) & \(H\) & \(I\) & \(G\) & \(I\) & \(G\) &
\(H\) & \(F\) \\
\bottomrule()
\end{longtable}

The following statements are true in Young's geometry: - each point is
incident with exactly four lines, - for every two distinct points there
are exactly five lines that are not incident with these points, and -
the Euclidean Parallel Property holds. \(\square\)

\hypertarget{hand-shake-geometry}{%
\subsection{Hand-Shake Geometry}\label{hand-shake-geometry}}

\begin{definition}[]\protect\hypertarget{def-hand-shake}{}\label{def-hand-shake}

A \textbf{hand-shake plane} is an incidence plane for which every line
has exactly two points.

\end{definition}

In the three examples above we see that a three (four, five) point
hand-shake geometry satisfies the Elliptic (Euclidean, Hyperbolic)
Parallel Property.

\begin{proposition}[]\protect\hypertarget{prp-three}{}\label{prp-three}

A hand-shake plane with more than five points satisfies the Hyperbolic
Parallel Property.

\end{proposition}

\begin{proof}

Suppose that \(l\) is a line and \(P\) is a point not incident with
\(l.\) Then there exists points \(A\) and \(B\) on line \(l\) that are
not \(P,\) say \(l=\{A,B\}.\) Further, since there are least five
points, there exists points \(Q\) and \(R\) that are not \(A,\) \(B,\)
nor \(P.\) Then lines \(m=\{P,Q\}\) and \(n=\{P,R\}\) are lines parallel
to \(l\) that are incident with \(P.\) \(\square\)

\end{proof}

\begin{theorem}[]\protect\hypertarget{thm-hand-shank-plane}{}\label{thm-hand-shank-plane}

In a hand-shank plane with \(n\) points, there are exactly
\(\frac{(n-1)n}{2}\) lines.

\end{theorem}

\begin{proof}

The hand-shake incidence geometry with \(n\) points has

\[1+2+\cdots + (n-2)+(n-1)= \frac{(n-1)n}{2}.\]

To see this we notice that we can connect the first point to the other
\((n-1)\) points. Disregarding this point, we connect the second point
to \((n-2)\) different points, and so on. The last line to be considered
is between the \((n-1)\)-th and the \(n\)-th point \(\square\)

\end{proof}

\hypertarget{straight-fan-geometry}{%
\subsection{Straight-Fan Geometry}\label{straight-fan-geometry}}

\begin{definition}[]\protect\hypertarget{def-straight-fan}{}\label{def-straight-fan}

A \textbf{straight-fan plane} is an incidence plane will all but one
point incident with exactly one line.

\end{definition}

\begin{definition}[]\protect\hypertarget{def-quadrilateral}{}\label{def-quadrilateral}

Four distinct points, no three of which are incident, are said to be a
\textbf{quadrangle}. Four distinct lines, no three of which are incident
with a point, are said to be a \textbf{quadrilateral}.

\end{definition}

\begin{theorem}[]\protect\hypertarget{thm-straight-fan-quadrilateral}{}\label{thm-straight-fan-quadrilateral}

An incident plane that satisfies the Elliptic Parallel Property is
either a straight-fan plane or a quadrangle exists.

\end{theorem}

\begin{proof}

We assume the Elliptic Parallel Property holds, but no quadrangle
exists. Notice that the three-point incidence geometry is a
straight-fan. We now assume at least four points exist, and so take any
four points of the given incidence plane. Since no quadrangle exists, we
can assume that the three points \(A,\) \(B,\) \(C\) are incident with
line \(l.\) By Theorem 4, there exists a point \(P\) not on the line
\(l.\)

Assume towards a contradiction that two distinct points \(Q\) both are
not incident with \(l.\) The lines \(l(PQ)\) and \(l\) are incident,
either in one of the three points \(A,\) \(B,\) \(C,\) or still another
point. WLOG assume the intersection point \(A'\) is different from both
\(B\) and \(C.\) Because no quadrangle exists, of the four points \(B,\)
\(C,\) \(P,\) \(Q\) at least three are incident with a line. This could
be points \(P,\) \(Q,\) \(B\) or \(P,\) \(Q,\) \(C.\) Both cases are
impossible, since the lines \(l(PQ)\) and \(l =l(BC)\) are incident only
in point \(A'\) which is not \(B\) nor \(C.\) Therefore, there exists
exactly one point \(P\) not incident with line \(l,\) as desired.
\(\square\)

\end{proof}

\begin{theorem}[]\protect\hypertarget{thm-straight-fan-plane}{}\label{thm-straight-fan-plane}

Every straight-fan plane satisfies the Elliptic Parallel Property.

\end{theorem}

\begin{proof}

In a straight-fan plane, all points except one are incident with one
line, call it line \(l.\) Clearly, every line must be incident with this
line, and so there are no parallel lines. It then follows that the
Elliptic Parallel Property must hold. \(\square\)

\end{proof}

\begin{theorem}[]\protect\hypertarget{thm-straight-fan-points-lines}{}\label{thm-straight-fan-points-lines}

In a straight-fan plane with \(n\) points, there are exactly \(n\)
lines.

\end{theorem}

\begin{proof}

Notice there is one line \(l\) with \(n-1\) points, and only one point
\(P\) not incident with \(l.\) There are \(n-1\) lines with two points
each of which connects \(P\) to a different point on the line \(l.\)
\(\square\)

\end{proof}

\hypertarget{introduction-to-affine-planes}{%
\section{Introduction to Affine
Planes}\label{introduction-to-affine-planes}}

Leonhard Euler, in 1748, introduced the term \textbf{affine} in his
(extremely) influential book \emph{Introductio in analysin infinitorum}
(volume 2, chapter XVIII).

\hypertarget{definitions-and-propositions}{%
\subsection{Definitions and
Propositions}\label{definitions-and-propositions}}

We will now switch out Incidence Axiom (A2) and replace it with the
Euclidean Parallel Property. By the way, sometimes this statement is
called Playfair's Axiom, hence the abbreviation (PA).

\begin{definition}[]\protect\hypertarget{def-affine-plane}{}\label{def-affine-plane}

An \textbf{affine plane} consists of a set of points, and a set of
lines, satisfying the axioms:

\end{definition}

\begin{itemize}
\tightlist
\item
  (A1) Every two distinct points are incident with a unique line.
\item
  (PA) (Euclidean Parallel Property) If a point \(P\) is not incident
  with a line \(l,\) there exists exactly one line \(m\) incident with
  \(P\) that is parallel to \(l.\)
\item
  (A3) There exists three distinct points that are not incident with the
  same line.
\end{itemize}

Surprise! In an affine plane, we get the Incidence Axiom (A2) back (for
free).

\begin{proposition}[Affine
Plane]\protect\hypertarget{prp-twodispointaff}{}\label{prp-twodispointaff}

In an affine plane, the Euclidean Parallel Property and the Incidence
Axioms hold.

\end{proposition}

\begin{proof}

It remains to prove (A2) that there are at least two distinct points on
every line. To do so, suppose that \(l\) is a line. By (A3) there exists
three distinct points \(A,\) \(B,\) and \(C\) that are not incident with
the same line. By (A1), there exists lines \(l(AB),\) \(l(AC),\) and
\(l(BC).\) If \(l\) is one of these lines, then \(l\) is incident with
two distinct points. Hence we assume that \(l\) is not any of these
lines. If \(l\) is incident with any two of these three points, then we
are finished.

Case 1: Assume that \(l\) is incident with at most one of these points,
say point \(A.\) Since \(A,\) \(B,\) \(C\) are noncollinear, \(B\) is
not incident with \(l(AC).\) By (PA) there exists unique line \(m\)
incident with \(B\) that is parallel to \(l(AC).\) Case 1.1: Suppose
\(l \parallel m\) and \(l \parallel l(BC).\) Since both \(m\) and
\(l(BC)\) pass through \(B\) and are parallel to \(l,\) (PA) yields
\(m=l(BC).\) So that \(C\) lies on \(m.\) However, \(m \parallel l(AC)\)
shows that this case can not happen. Case 1.2: Suppose
\(l\not\parallel m.\) So \(l\) and \(m\) have a point in common, say
\(D.\) If \(A=D,\) then by (A1), \(m=l(AB).\) However, since
\(m \parallel l(AC)\) we can not have \(m=l(AB).\) Thus \(A\neq D,\) and
so \(l\) has two distinct points on it. Case 1.3: Suppose
\(l\not\parallel l(BC).\) So \(l\) and \(l(BC)\) have a point in common,
say \(D.\) Notice that if \(A=D,\) then \(A, B, C\) are collinear to
\(l(BC).\) Thus \(A\neq D,\) and so \(l\) has two distinct points on it.
We conclude case 1 with: line \(l\) has two distinct points on it. (See
\eqref{fig:twopoints})

Case 2: Assume that \(A,\) \(B,\) and \(C\) are not incident with \(l.\)
Case 2.1: Suppose that \(l \parallel l(BC).\) If \(l allell(AB),\) then
\(l(BC)=l(AB)\) by (P4). However, this case can not happen because
\(A,\) \(B,\) and \(C\) are noncollinear. Hence \(l(AB)\) and \(l\) are
incident with a point, say \(P.\) If \(l allell(AC),\) then \(l(BC)\)
and \(l(AC)\) by (P4). However, this case can not happen because \(A,\)
\(B,\) and \(C\) are noncollinear. Hence \(l(AC)\) and \(l\) are
incident with a point, say \(Q.\) Notice that \(P\neq A\) and
\(Q\neq A\) since \(P, Q\) are incident with \(l\) and \(A\) is not.
Hence, if \(P=Q,\) then \(l(AC)=l(AB)\) contrary to the assumption that
\(A,\) \(B,\) and \(C\) are noncollinear. Whence \(l\) is incident with
distinct points \(P\) and \(Q.\) The case 2.2 for \(l \parallel l(AC)\)
and case 2.3 for \(l \parallel l(AB)\) are entirely similar. Case 2.4:
Now suppose that \(l\) is not parallel to any of these lines. Then
\(l(AB)\) and \(l\) are incident at \(P\) and \(l(AC)\) are incident at
\(Q.\) As we found above \(P\neq Q,\) and so again \(l\) is incident
with distinct points \(P\) and \(Q.\) \(\square\)

\end{proof}

\textbf{Remark}. By \textbf{Affine Plane}, it follows that Theorem 1
through Theorem 11 are theorems in any affine plane.

\begin{proposition}[]\protect\hypertarget{prp-five}{}\label{prp-five}

In an affine plane, if a line is incident with, and distinct from, one
of two distinct parallel lines, then it is also incident with the other.

\end{proposition}

\begin{proof}

Let \(l, m\) and \(n\) be three distinct lines. Assume that
\(l \parallel m\) and line \(n\) is incident with \(m.\) By Theorem 1,
there is only one point \(P\) incident with both \(n\) and \(m.\) If
\(l \parallel n,\) then \(n=m\) by (P4). Hence \(l\not \parallel n\) and
so \(n\) must meet line \(l.\) \(\square\)

\end{proof}

\begin{proposition}[]\protect\hypertarget{prp-five}{}\label{prp-five}

\label{parequiv} In an affine plane, parallelism is an equivalence
relation.

\end{proposition}

\begin{proof}

By \textbf{Parallelism is Transitive} and
\eqref{twodispointaff}.\(\square\)

\end{proof}

\begin{definition}[]\protect\hypertarget{def-pencil}{}\label{def-pencil}

A \textbf{pencil} is a set consisting of a line, together with all lines
parallel to it, that is, an equivalence class of parallel lines under
parallelism.

\end{definition}

\begin{proposition}[]\protect\hypertarget{prp-five}{}\label{prp-five}

\label{threepencils} In an affine plane, there are at least three
pencils of parallel lines.

\end{proposition}

\begin{proof}

There exists three noncollinear points, say \(A,\) \(B,\) and \(C.\) As
in Theorem 2 , \(l(AB),\) \(l(AC),\) and \(l(BC)\) are three distinct
nonconcurrent lines. By (PA), any line \(l\) parallel to \(l(AB)\) must
intersect \(l(AC),\) and \(l(BC)\) because \(A,\) \(B,\) and \(C\) are
noncollinear. Hence \(l\) is only in the pencil corresponding to
\(l(AB).\) Similarly for the other cases. Therefore, there are at least
three distinct pencils of parallel lines. \(\square\)

\end{proof}

\begin{proposition}[]\protect\hypertarget{prp-bijection}{}\label{prp-bijection}

In an affine plane, all lines have the same cardinality.

\end{proposition}

\begin{proof}

Assume \(l\neq m\) since otherwise the result is obvious. By Theorem 1 ,
it follows that either \(l\) and \(m\) have a unique point \(O\) in
common (case 1) or that \(l\) and \(m\) are parallel (case 2). (See
\eqref{fig:bijection})

Case 1: There exists points \(A\) and \(A'\) on \(l\) and \(m,\)
respectively, such that \(A\neq O\) and \(A'\neq O\); and there exists a
line \(l(AA').\) Let \(X\) be a point on \(l\) different from \(A\) and
\(O.\) There exists a unique line \(k\) through \(X\) and parallel to
\(l(AA').\) Suppose that \(k \parallel m.\) Then by (P4), \(m=l(AA')\)
since both lines \(m\) and \(l(AA')\) pass through \(A'\) and are
parallel to \(k.\) Hence \(k \parallel m\) can not happen. By Theorem 1,
it follows that \(k\) intersects \(m\) at a unique point, which we will
call \(X'.\) Thus it is easy to see that the function \(f:l\to m\)
prescribed by \(f(O)=O,\) \(f(A)=A',\) and \(f(X)=X'\) is well-defined.

To show \(f\) is a surjective assume point \(C\) lies on \(m.\) If
\(C=O\) or \(C=A',\) then \(f(O)=C\) or \(f(A)=C\) as needed,
respectively. Assume \(C\) is distinct from \(O\) and \(A'.\) By (P4),
there exists a line \(k'\) passing through \(C\) and parallel to
\(l(AA').\) Suppose that \(k' \parallel l.\) Then \(l\) and \(l(AA')\)
pass through \(A\) and are parallel to \(k'\) so by (P4), \(l=l(AA'),\)
which can not happen. Hence it follows that \(k'\) can not be parallel
to \(l,\) and so \(k'\) intersects \(l\) at a single point \(C'.\) Then
it follows \(f(C')=C,\) and so \(f\) is surjective.

Now assume there exists points \(B\) and \(C\) on \(l\) such that
\(f(B)=f(C)=D.\) Since both \(l(BB')\) and \(l(CC')\) are parallel to
\(l(AA')\) and both contain \(D,\) we have \(l(BB')=l(CC'):=k''\) by
(P4). By Theorem 1, \(k''\) and \(l\) have a unique point in common, so
\(B=C\) as needed.\\
Therefore, \(f\) is one-to-one.

Case 2: Assume \(l \parallel m.\) Let \(O\) lie on \(l\) and \(O'\) lie
on \(m.\) It follows that \(l\) and \(l(OO')\) are incident at \(O,\) so
by case 1 there is a bijection \(f\) from \(l\) to \(l(OO').\) Similarly
there is a bijection \(g\) between \(l(OO')\) and \(m.\) Therefore
\(g\circ f\) is a bijection of \(l\) and \(m.\) \(\square\)

\end{proof}

\hypertarget{finite-affine-planes}{%
\subsection{Finite Affine Planes}\label{finite-affine-planes}}

A \textbf{finite affine plane} is an affine plane with a finite number
of points. A finite affine plane with \(n^2\) points is said to have
\textbf{order} \(n.\)

\begin{proposition}[]\protect\hypertarget{prp-five}{}\label{prp-five}

In a finite affine plane, there exists an integer \(n\geq 2\) such that

\begin{itemize}
\tightlist
\item
  the total number of points is exactly \(n^2,\)
\item
  every point is incident with exactly \(n+1\) lines,
\item
  every pencil contains exactly \(n\) lines,
\item
  there are exactly \(n^2+n\) lines, and
\item
  there are exactly \(n+1\) pencils of parallel lines.
\end{itemize}

\end{proposition}

\begin{proof}

\eqref{fapone}: There is at least one line, and by \eqref{Bijection},
there are \(n\) points on it, say \(l=\{A_1, \dots, A_n\}.\) There
exists a point \(B\) not on \(l\) and consider \(l(A_1B)=m_1.\) For
\(i=2, \ldots, ,n\) we use the Euclidean Parallel Property to define
\(m_i\) to be the unique line through \(A_i\) parallel to \(m_1.\) Each
of these lines \(m_1,\ldots, , m_n\) has \(n\) points, there are \(n\)
of these lines, and no pair of lines has a point in common. Therefore,
the number of points in \(\mathcal{P}\) is at least \(n^2.\) If \(P\) is
any point in \(\mathcal{P},\) then either \(P\) is on \(m_1\) or there
is a line \(k\) through \(P\) parallel to \(m_1.\) In the latter case
\(k\) must intersect \(l\) at one of the points \(A_2, \ldots, A_n,\)
because if not then \(l \parallel k,\) \(m_1 \parallel k,\) and both of
them passes through \(A_1\) means \(l=m_1.\) Therefore, we have
\(k=m_i\) for some \(i.\) Thus every point in \(\mathcal{P}\) lies on
one of the lines \(m_i,\) and in fact \(\mathcal{P}\) has exactly
\(n^2\) points.

\eqref{faptwo}: Suppose that \(B\) is an arbitrary point, and
\(l=\{A_1,\ldots, A_n\}\) is a line not containing \(B.\) By (A1) and
Theorem 1, the lines \(l_i=l(BA_i)\) (for \(i=1,\ldots,n)\) and the line
\(l_{n+1},\) the line through \(B\) parallel to \(l\) (by (PA)), are all
distinct. Further, any line through \(B\) intersects \(l\) (and
therefore is one of the \(l_i\)) or is parallel to \(l\) (and therefore
is \(l_{n+1}\)). (See \eqref{fig:pencil})

\eqref{fapthree}: Suppose that \(l\) is any line, \(C_1\) is a point on
\(l,\) and \(C_2\) is a point not on \(l.\) Then the points of the line
\(l(C_1C_2)\) can be written as \(C_1, \ldots, C_n.\) For
\(i=2, \ldots, n,\) let \(l_i\) be the line through \(C_i\) parallel to
\(l,\) as shown in \eqref{fig:pencil}. These \(n-1\) lines are all
different, since the points \(C_2, \dots, C_n\) are distinct. If \(k\)
is any line parallel to \(l,\) then \(k\) must intersect \(l(C_1C_2),\)
since \(l\) is the only line containing \(C_1\) which is parallel to
\(k.\) Therefore \(k\cap l(C_1C_2)=C_i\) for some \(i,\) so \(k=l.\)
Consequently the lines \(l_2,\ldots, l_n\) are the only lines parallel
to \(l.\) (See \eqref{fig:pencil})

\eqref{fapfour}: Finally, suppose that \(B\) is any point, and
\(l_1,\ldots, l_{n+1}\) are the \(n+1\) lines containing \(B.\) Each of
these lines is in a pencil consisting of \(n\) lines, so the lines
through \(B,\) together with the lines parallel through them, account
for \(n(n+1)\) distinct lines. For any line \(m,\) either \(B\) lies on
\(m\) (in which case \(m\) has been counted) or there is a line
containing \(B\) and parallel to \(m\) (in which case \(m\) has also
been counted). Therefore there are \(n(n+1)\) lines in \(\mathcal{L}.\)

\eqref{fapfive}: Since each of the \(n(n+1)\) lines is in one and only
one pencil, and each pencil contains \(n\) lines, there are \(n+1\)
pencils. \(\square\)

\end{proof}

\hypertarget{introduction-to-projective-planes}{%
\section{Introduction to Projective
Planes}\label{introduction-to-projective-planes}}

Now it's our turn in investigate the Elliptic Parallel Property more.

\hypertarget{definitions-and-propositions-1}{%
\subsubsection{Definitions and
Propositions}\label{definitions-and-propositions-1}}

\begin{definition}[]\protect\hypertarget{def-projective-plane}{}\label{def-projective-plane}

A \textbf{projective plane} consists of a set of points, and a set of
lines, satisfying the axioms:

\begin{itemize}
\tightlist
\item
  (P1) Every two distinct points (lines) are incident with a unique line
  (point).
\item
  (P2) There exists four distinct points no three of which are
  collinear.
\end{itemize}

\end{definition}

\begin{proposition}[]\protect\hypertarget{prp-twodispointpro}{}\label{prp-twodispointpro}

In a projective plane, the Elliptic Parallel Property and the Incidence
Axioms hold.

\end{proposition}

\begin{proof}

To prove that the Elliptic Parallel Property holds assume that \(l\) is
a line and \(P\) is a point not incident with \(l.\) Suppose that \(m\)
is a line parallel to \(l\) that is incident with \(P.\) If \(l=m,\)
then \(P\) is on \(l\) contary to hypothesis. Hence \(l\neq m\) and then
by (P1), \(l\) and \(m\) have point in common. Hence \(m\) does not
exist and so the Elliptic Parallel Property holds.

By (P1) every two distinct points determines a unique line, and so (A1)
holds.

By (P2) there exists four distinct points, say \(A,\) \(B,\) \(C,\) and
\(D,\) no three of which are collinear. In particular, \(A,\) \(B,\) are
\(C\) noncollinear, and so (A3) holds.

To prove that (A2) holds, assume that \(l\) is a line. By (P2), there
exists four distinct points \(A,\) \(B,\) \(C,\) and \(D,\) no three of
which are collinear. By (P1), there exists a line incident with \(A\)
and \(B,\) say \(l(AB).\) By (P1), there exists a line incident with
\(A\) and \(C,\) say \(l(AC).\) If \(l=l(AB)\) or \(l=l(AC),\) then
\(l\) is incident with two distinct point as needed. Assume that
\(\neq l(AB)\) and \(l\neq l(AC).\) By (P1), there exists points
\(E, F\) incident with \(l.\) If \(E=F,\) then \(l(AB)=l(AC)\) by (P1),
which can not happen. Whence \(E\neq F\) and so (A2) holds. \(\square\)

\end{proof}

\textbf{Remark}. By (Proposition~\ref{prp-twodispointpro}), it follows
that Theorem 1 through Theorem 11 are propositions in any projective
plane.

\begin{proposition}[]\protect\hypertarget{prp-threecon}{}\label{prp-threecon}

In a projective plane, there exist four distinct lines no three of which
are concurrent.

\end{proposition}

\begin{proof}

By (P2) there exists four distinct points, \(A,\) \(B,\) \(C,\) and
\(D\) no three of which are collinear.\\
By (P1) there exists lines \(l(AB),\) \(l(BC),\) \(l(CD),\) and
\(l(DA).\) Assume that three of them are concurrent, WLOG, say
\(l(AB),\) \(l(AD),\) and \(l(CD).\) By definition of concurrent and by
(P1), there exists a unique point in common to these lines, say \(E.\)
Now \(A\) is incident \(l(AB),\) \(l(AD)\) and so by Theorem 1 ,
\(E=A.\) Similarly, \(D\) is incident \(l(AD),\) \(l(CD)\) and so by
Theorem 1 , \(E=D.\) Now since \(A=D\) is contrary to hypothesis, we see
that no three of these lines can be concurrent. \(\square\)

\end{proof}

\begin{proposition}[]\protect\hypertarget{prp-threepoints}{}\label{prp-threepoints}

In a projective plane, every line is incident with at least three
points.

\end{proposition}

\begin{proof}

Suppose there exists a line \(l\) incident with only two points, say
\(l=\{P,Q\}.\) By \eqref{threecon}, there exists four distinct points
\(A,\) \(B,\) \(C,\) and \(D\) no three of which are collinear and there
exists four distinct lines \(l(AB),\) \(l(BC),\) \(l(CD),\) and
\(l(DA)\) no three of which are concurrent. By (P1) each of these four
lines are incident with \(l\) and this must occur at \(P\) or \(Q\) by
hypothesis. Since no three of these lines are concurrent, two of these
lines, say \(l(AB)\) and \(l(CD)\) are incident with \(l\) at \(P\) and
two of these lines, say \(l(BC)\) and \(l(DA)\) are incident with \(l\)
at \(Q.\) By (P1), there exists a line \(l(AC)\) and is incident with
\(l.\) If \(l(AC)\) is incident with \(l\) at \(P,\) then \(A,\) \(B,\)
and \(C\) are collinear because \(l(AC)=l(AB)\) by (P1). If \(l(AC)\) is
incident with \(l\) at \(Q,\) then \(A,\) \(B,\) and \(C\) are collinear
because \(l(AC)=l(AB)\) by (P1). Whence our hypothesis of only two
points on \(l\) is a contradiction and so the result follows by
\eqref{twodispointpro}. \(\square\)

\end{proof}

\begin{example}[]\protect\hypertarget{exm-fanoplane}{}\label{exm-fanoplane}

~

\begin{itemize}
\tightlist
\item
  There are exactly seven lines where each line consists of exactly
  three points.
\item
  The Elliptic Parallel Property holds.
\item
  Each point is incident with exactly three lines.
\item
  For every two distinct points there are exactly two lines that are not
  incident with these points.
\end{itemize}

\begin{longtable}[]{@{}lllllll@{}}
\toprule()
\(l_1\) & \(l_2\) & \(l_3\) & \(l_4\) & \(l_5\) & \(l_6\) & \(l_7\) \\
\midrule()
\endhead
A & A & A & B & B & C & C \\
B & D & F & D & E & D & E \\
C & E & G & F & G & G & F \\
\bottomrule()
\end{longtable}

\end{example}

\begin{theorem}[]\protect\hypertarget{thm-neitherone}{}\label{thm-neitherone}

In a projective plane, for any two distinct lines there exist a point
not incident with either of them.

\end{theorem}

\begin{proof}

Assume on the contrary that every point is incident with two distinct
lines \(l\) and \(m.\) There exists four distinct points \(A,\) \(B,\)
\(C,\) and \(D\) no three of which are collinear. Now since any three of
these points are not incident with the same line but all points are
incident with either \(m\) or \(n,\) it follows that two of these
points, say \(A\) and \(B\) are incident with \(l\) and the others are
incident with \(m.\) By (P1), lines \(l(AC)\) and \(l(BD)\) are incident
with a unique point, say \(X.\)

Suppose \(X\) is incident with \(l.\) Since \(l\) is incident with \(A\)
and \(B,\) \(l=l(AB).\) Thus \(X\) is incident with \(l(AB).\) Since
\(A,\) \(B,\) and \(C\) are noncollinear, \(l(AB)\neq l(AC)\) and thus,
since \(X\) is incident with \(l(AC),\) \(X\) must be the unique point
incident with lines \(l(AB)\) and \(l(AC).\) But \(A\) is also incident
with both these lines. Thus \(X=A.\) However, this is impossible since
\(X\) is incident with the line \(l(BD)\) and \(A,\) \(B,\) and \(D\)
are noncollinear. Hence \(X\) is not incident with \(l.\) The case for
\(X\) is not incident with \(m\) is similar. \(\square\)

\end{proof}

\begin{proposition}[]\protect\hypertarget{prp-probijection}{}\label{prp-probijection}

In a projective plane, all lines have the same cardinality.

\end{proposition}

\begin{proof}

Suppose that \(l\) and \(m\) are lines. By
(Theorem~\ref{thm-neitherone}), there is a point \(O\) not incident with
either line.

For any point \(X\) incident with \(l\) prescribe a point \(X^*\)
incident with \(m\) using that \(O\neq X\) and \(l(OX)\) is incident
with a unique point \(X^*\) on \(m\) because \(l(OX)\neq m.\) Similarly,
for any point \(Y\) incident with \(m\) prescribe a point \(Y^*\)
incident with \(l\) using that \(O\neq Y\) and \(l(OY)\) is incident
with a unique point \(Y^*\) on \(l\) because \(l(OY)\neq l.\)

Suppose that \(X\) is incident with \(l,\) that \(X^*\) is incident with
\(m,\) and that \(X^{**}\) is incident with \(l\) prescribed above,
respectively. It remains to show that \(X=X^{**}.\) If \(X\neq X^{**},\)
then \(l(OX)=l(XX^{**})=l(OX^{**})=l\) contrary to the hypothesis that
\(O\) is not incident with \(l.\) \(\square\)

\end{proof}

\begin{theorem}[]\protect\hypertarget{thm-biptln}{}\label{thm-biptln}

In a projective plane, if \(l\) is any line and \(P\) is any point, then
there is a bijection between the points incident with \(l\) and the
lines incident with \(P.\)

\end{theorem}

\begin{proof}

Suppose that \(P\) is a point and \(l\) is a line. For case one, assume
\(P\) is not incident with \(l.\) In the proof of \eqref{probijection},
a bijection between the points incident with \(l\) and lines incident
with \(P\) is proven to exist.

For a second case, assume that \(P\) is incident with \(l.\) By
\eqref{threecon}, there exists a line \(m\) such that \(P\) is not
incident with \(m.\) By case one, there is a bijection between the lines
incident with \(P\) and the points of \(m.\)\\
By \eqref{probijection}, there is a bijection between the points of
\(m\) and the points of \(l.\) The composition of these bijections is
the desired one. \(\square\)

\end{proof}

\hypertarget{finite-projective-planes}{%
\subsubsection{Finite Projective
Planes}\label{finite-projective-planes}}

A \textbf{finite projective plane} is a projective plane with a finite
number of points. A finite projective plane with \(n^2+n+1\) points is
said to have \textbf{order} \(n.\)

\begin{proposition}[]\protect\hypertarget{prp-fpp}{}\label{prp-fpp}

In a finite projective plane, there exists an integer \(n\geq 2\) such
that

\begin{enumerate}
\def\labelenumi{\arabic{enumi}.}
\tightlist
\item
  every point (line) is incident with exactly \(n+1\) lines (points),
  and
\item
  there are exactly \(n^2+n+1\) points (lines).
\end{enumerate}

\end{proposition}

\begin{proof}

By (Theorem~\ref{thm-biptln}), all lines have the same number of points,
call this number \(n+1.\)

(1): Let \(P\) be a point. By Theorem 5 , there exists aline \(l\) not
incident with \(P.\) Let \(Q_1, \ldots, Q_{n+1}\) be the points incident
with \(l.\) If \(l(PQ_1)\) and \(i\neq j,\) then \(P, Q_i, Q_j\) are
distinct. So we have \(n+1\) lines incited with \(P,\) and so a
contradiction. The lines \(l(PQ_1), \ldots, l(PQ_{n+1)}\) are distinct.
So we have \(n+1\) lines incident with \(P.\) Now suppose that \(m\) is
any line indent with \(P.\) Then \(m\) must be incident with \(l,\) say
at \(Q_i.\) Then \(P\) and \(Q_i\) both are incident with \(m\) and
\(l(PQ_1).\) This show that there are exactly \(n+1\) lines through
\(P.\)

(2): Let \(P\) be a point. By \eqref{fppone}, there are \(n+1\) lines
incident with \(P.\) Since any two points are incident with a unique
line, every point except \(P\) is incident with exactly one of these
lines. By \eqref{fppone}, each of these lines is incident with \(n\)
points distinct from \(P.\) Thus the total number of points is
\((n+1)n+1\) as needed. \(\square\)

\end{proof}

\hypertarget{affine-to-projective}{%
\subsection{Affine to Projective}\label{affine-to-projective}}

Suppose that \((\mathcal{P},\mathcal{L})\) is an affine plane. For each
pencil \(\phi\) of parallel lines, prescribe a symbol \(X_\phi.\) The
symbols \(X_\phi\) are called \textbf{points at infinity}. Let \[
\mathcal{P}'
=\mathcal{P}\cup 
\{X_\phi \mid \phi \text{ is a pencil of parallel lines in $\mathcal{L}$}\}. 
\] For each \(l\in \mathcal{L},\) prescribe \(l'=l\cup \{X_\phi\},\)
where \(\phi\) is the unique pencil containing \(l.\) Let \[
l_{\infty}=\{X_{\phi} \mid \phi \text{ is a pencil of a parallel lines in $\mathcal{L}$} \}.
\] Finally, prescribe
\(\mathcal{L}'=\{l' \mid l \in \mathcal{L}\}\cup \{l_{\infty}\}.\)

\begin{theorem}[]\protect\hypertarget{thm-}{}\label{thm-}

If \((\mathcal{P},\mathcal{L})\) is an affine plane, then
\((\mathcal{P}',\mathcal{L}')\) is a projective plane.

\end{theorem}

\begin{proof}

(P1): Let \(P\) and \(Q\) be two disinfect points in \(\mathcal{P}'.\)
If \(P, Q\in \mathcal{P},\) then there is unique (affine) line \(l\)
incident with \(P\) and \(Q,\) therefore the line \(l'\) is the unique
projective line incident with them. If \(P\in \mathcal{P}\) and
\(Q=X_{\phi},\) then \(P\) is incident with a unique affine line \(l\)
in pencil \(\phi,\) and so the unique projective line incident with
\(P\) and \(Q\) is \(l'.\) The unique line incident with two points
\(X_\phi\) and \(X_\theta\) is \(l_\infty.\)

(P1): Notice that lines of the form \(l'\) and \(m'\) were either formed
by augmenting intersecting affine lines \(l\) and \(m,\) in which case
they still intersect as projective lines, or they were formed from
parallel lines \(l\) and \(m,\) contained in some pencil \(\phi.\) Thus
the lines \(l'\) and \(m'\) intersect in \(X_\phi.\) The line \(l'\)
contains \(X_\phi\) where \(\phi\) is the pencil containing \(l\); thus
\(l'\cap l_{\infty}=X_\phi.\)

(P2): Since there are at least two points incident with every line
\(l,\) there are at least three points on every projective line \(l'.\)
By \eqref{threepencils}, every affine plane contains at least three
pencils of parallel lines, \(l_{\infty}\) contains at least three
points. By Theorem 2 , there are at least two lines in every affine
plane, there are at least two lines in \(\mathcal{L}'.\) Hence there are
at least three points on each line; there are at least two lines.
Therefore (P2) must hold. \(\square\)

\end{proof}

\begin{example}[]\protect\hypertarget{exm-four-oints}{}\label{exm-four-oints}

Recall the affine plane from \textbf{4 Points} using
\(\mathcal{P}=\{A, B, C, D\}\) and
\(\mathcal{L}=\{l_1, l_2, l_3, l_4, l_5, l_6\}.\)

\begin{longtable}[]{@{}llllll@{}}
\toprule()
\(l_1\) & \(l_2\) & \(l_3\) & \(l_4\) & \(l_5\) & \(l_6\) \\
\midrule()
\endhead
A & A & A & B & B & C \\
B & C & D & C & D & D \\
\bottomrule()
\end{longtable}

We can embed this plane \((\mathcal{P},\mathcal{L})\) into a projective
plane \((\mathcal{P}',\mathcal{L}').\) First we determine the pencils
\([l_1]=\{l_1, l_6\},\) \([l_2]=\{l_2, l_5\},\) \([l_3]=\{l_3, l_4\}.\)
We set \(\mathcal{P}'=\mathcal{P}\cup \{X_1, X_2, X_3\}\) and we form
the lines for the new plane:

\begin{longtable}[]{@{}lllllll@{}}
\toprule()
\(l_1\) & \(l_2\) & \(l_3\) & \(l_4\) & \(l_5\) & \(l_6\) &
\(l_{\infty}\) \\
\midrule()
\endhead
\(X_1\) & A & A & B & B & \(X_1\) & \(X_1\) \\
A & C & D & C & D & C & \(X_2\) \\
B & \(X_2\) & \(X_3\) & \(X_3\) & \(X_2\) & D & \(X_3\) \\
\bottomrule()
\end{longtable}

\end{example}

\hypertarget{projective-to-affine}{%
\subsection{Projective to Affine}\label{projective-to-affine}}

Removing a line and all its points from a projective plane results in an
affine plane. Specifically, suppose we are given a projective plane
\((\mathcal{P}',\mathcal{L}').\) Choose any line from \(\mathcal{L},\)
and call it \(l_{\infty}.\) Prescribe \[
\mathcal{P}=\{P\mid P\in \mathcal{P}', P\notin l_{\infty}\}
=\mathcal{P}'\setminus l_{\infty}.
\] For each line \(l'\neq l_{\infty},\) prescribe \[
l=\{P\mid P\in l', P\notin l_{\infty}\}=l'\setminus (l'\cap l_{\infty}).
\] Let \(\mathcal{L}=\{l\mid l'\in \mathcal{L}', l'\neq l_{\infty}\}.\)

\begin{theorem}[]\protect\hypertarget{thm-projective-plane}{}\label{thm-projective-plane}

If \((\mathcal{P}',\mathcal{L}')\) is a projective plane, then
\((\mathcal{P},\mathcal{L})\) is an affine plane.

\end{theorem}

\begin{proof}

(see \textcite{MR1344447}) Any two distinct points of \(\mathcal{P}'\)
are incident with a unique line \(l'.\) If the points remains in
\(\mathcal{P},\) they are still incident with line \(l,\) and only
incident with line \(l,\) so \ref{A1} holds. Suppose that
\(l'\cap l_{\infty}=Q\in \mathcal{P}'.\) Let \(m'=l(PQ).\) Then
\(l'\cap m'=Q.\) Thus \(l\cap m\) is empty, and \(m\) is the unique line
containing \(P\) and parallel to \(l.\) Hence \ref{A2} holds. Since each
line in \(\mathcal{L}'\) contains at least three points, each line in
\(\mathcal{L}\) contains at least two points. Since there are at least
three lines in \(\mathcal{L}',\) there are at least two lines remaining
in \(\mathcal{L},\) and so \ref{A3} holds. Thus
\((\mathcal{P},\mathcal{L})\) is an affine plane. \(\square\)

\end{proof}


\printbibliography



\thispagestyle{empty}


\end{document}
