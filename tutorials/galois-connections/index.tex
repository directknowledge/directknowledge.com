% Options for packages loaded elsewhere
\PassOptionsToPackage{unicode}{hyperref}
\PassOptionsToPackage{hyphens}{url}
%
\documentclass[
  twoside,
  12pt,
  letterpaper,
  fleqn]{article}

\usepackage{amsmath,amssymb}
\usepackage{lmodern}
\usepackage{iftex}
\ifPDFTeX
  \usepackage[T1]{fontenc}
  \usepackage[utf8]{inputenc}
  \usepackage{textcomp} % provide euro and other symbols
\else % if luatex or xetex
  \usepackage{unicode-math}
  \defaultfontfeatures{Scale=MatchLowercase}
  \defaultfontfeatures[\rmfamily]{Ligatures=TeX,Scale=1}
\fi
% Use upquote if available, for straight quotes in verbatim environments
\IfFileExists{upquote.sty}{\usepackage{upquote}}{}
\IfFileExists{microtype.sty}{% use microtype if available
  \usepackage[]{microtype}
  \UseMicrotypeSet[protrusion]{basicmath} % disable protrusion for tt fonts
}{}
\makeatletter
\@ifundefined{KOMAClassName}{% if non-KOMA class
  \IfFileExists{parskip.sty}{%
    \usepackage{parskip}
  }{% else
    \setlength{\parindent}{0pt}
    \setlength{\parskip}{6pt plus 2pt minus 1pt}}
}{% if KOMA class
  \KOMAoptions{parskip=half}}
\makeatother
\usepackage{xcolor}
\setlength{\emergencystretch}{3em} % prevent overfull lines
\setcounter{secnumdepth}{5}
% Make \paragraph and \subparagraph free-standing
\ifx\paragraph\undefined\else
  \let\oldparagraph\paragraph
  \renewcommand{\paragraph}[1]{\oldparagraph{#1}\mbox{}}
\fi
\ifx\subparagraph\undefined\else
  \let\oldsubparagraph\subparagraph
  \renewcommand{\subparagraph}[1]{\oldsubparagraph{#1}\mbox{}}
\fi


\providecommand{\tightlist}{%
  \setlength{\itemsep}{0pt}\setlength{\parskip}{0pt}}\usepackage{longtable,booktabs,array}
\usepackage{calc} % for calculating minipage widths
% Correct order of tables after \paragraph or \subparagraph
\usepackage{etoolbox}
\makeatletter
\patchcmd\longtable{\par}{\if@noskipsec\mbox{}\fi\par}{}{}
\makeatother
% Allow footnotes in longtable head/foot
\IfFileExists{footnotehyper.sty}{\usepackage{footnotehyper}}{\usepackage{footnote}}
\makesavenoteenv{longtable}
\usepackage{graphicx}
\makeatletter
\def\maxwidth{\ifdim\Gin@nat@width>\linewidth\linewidth\else\Gin@nat@width\fi}
\def\maxheight{\ifdim\Gin@nat@height>\textheight\textheight\else\Gin@nat@height\fi}
\makeatother
% Scale images if necessary, so that they will not overflow the page
% margins by default, and it is still possible to overwrite the defaults
% using explicit options in \includegraphics[width, height, ...]{}
\setkeys{Gin}{width=\maxwidth,height=\maxheight,keepaspectratio}
% Set default figure placement to htbp
\makeatletter
\def\fps@figure{htbp}
\makeatother


\usepackage{geometry}

\geometry{reset, letterpaper, height=9in, width=6in, hmarginratio=1:1, vmarginratio=1:1, marginparsep=0pt, marginparwidth=0pt, headheight=15pt}

\usepackage{lipsum}

\def\dklogo{\raisebox{-.2\height}{\includegraphics[width=12pt]{../../assets/direct-knowledge-logo-book.png}}\footnotesize{\ Direct Knowledge}}
\usepackage{fancyhdr}
\pagestyle{fancy}
\fancyhead[LE]{\thepage}%
\fancyhead[RE]{\dklogo}%
\fancyhead[CE]{}%
\fancyfoot[LE]{}%
\fancyfoot[RE]{}%
\fancyfoot[CE]{}%
\fancyhead[LO]{D. A. Smith}%
\fancyhead[RO]{\thepage}%
\fancyhead[CO]{}%
\fancyfoot[LO]{}%
\fancyfoot[RO]{}%
\fancyfoot[CO]{}%


\renewcommand{\headrulewidth}{0.4pt}
\renewcommand{\footrulewidth}{0pt}

\usepackage{amssymb}
\usepackage{amsmath}
\usepackage{amsthm}
\usepackage{graphicx} 


\usepackage[most]{tcolorbox} 
\definecolor{block-gray}{gray}{0.97}
\newtcolorbox{zitat}[1][]{%
    colback=block-gray,
    grow to right by=-10mm,
    grow to left by=-10mm, 
    boxrule=0pt,
    boxsep=0pt,
    breakable,
    enhanced jigsaw,
    borderline west={2pt}{0pt}{gray},
    colbacktitle={block-gray},
    coltitle={black},
    fonttitle={\large\bfseries},
    attach title to upper={},
    #1,
}
\renewcommand{\quote}{\zitat}
\renewcommand{\endquote}{\endzitat}

\def\banner{ {\rule{\linewidth}{0.2pt}}
    \begin{minipage}[c]{14px}\includegraphics[width=14pt]{../../assets/direct-knowledge-logo-book.png}\end{minipage}
    \begin{minipage}[c]{150px}Direct Knowledge | Tutorials \end{minipage}
}

\usepackage{titling}
\setlength{\droptitle}{-8ex}
\pretitle{\vspace{-20pt}\banner\vspace{10pt}\begin{flushleft}\huge\bfseries}
\posttitle{\par\end{flushleft}}
\preauthor{\begin{flushleft}}
\postauthor{,
    \small{Direct Knowledge, USA}
    \footnote{email: david@directknowledge.com}
    \footnote{\copyright \, 2023 \ David A. Smith}
    \footnote{With authorization, you can freely share and reproduce portions of this work for educational or personal use. Please note that distributing any portion of it in print form requires further permission from its original authors, as does posting online to public servers or mailing lists without prior consent.}
    \end{flushleft}
    }
\predate{\begin{flushleft}}
\postdate{\end{flushleft}}

\renewenvironment{abstract}
{\par\noindent\textbf{\abstractname.}\ \ignorespaces \itshape}
{\par\medskip}


\usepackage{setspace}
\linespread{1.25}
\AtEndEnvironment{solution}{\vspace{-1.5\baselineskip}\hfill\qedsymbol}
\AtEndEnvironment{proof}{\vspace{-1.25\baselineskip}}




\makeatletter
\makeatother
\makeatletter
\makeatother
\makeatletter
\@ifpackageloaded{caption}{}{\usepackage{caption}}
\AtBeginDocument{%
\ifdefined\contentsname
  \renewcommand*\contentsname{Table of contents}
\else
  \newcommand\contentsname{Table of contents}
\fi
\ifdefined\listfigurename
  \renewcommand*\listfigurename{List of Figures}
\else
  \newcommand\listfigurename{List of Figures}
\fi
\ifdefined\listtablename
  \renewcommand*\listtablename{List of Tables}
\else
  \newcommand\listtablename{List of Tables}
\fi
\ifdefined\figurename
  \renewcommand*\figurename{Figure}
\else
  \newcommand\figurename{Figure}
\fi
\ifdefined\tablename
  \renewcommand*\tablename{Table}
\else
  \newcommand\tablename{Table}
\fi
}
\@ifpackageloaded{float}{}{\usepackage{float}}
\floatstyle{ruled}
\@ifundefined{c@chapter}{\newfloat{codelisting}{h}{lop}}{\newfloat{codelisting}{h}{lop}[chapter]}
\floatname{codelisting}{Listing}
\newcommand*\listoflistings{\listof{codelisting}{List of Listings}}
\usepackage{amsthm}
\theoremstyle{definition}
\newtheorem{example}{Example}[section]
\theoremstyle{definition}
\newtheorem{definition}{Definition}[section]
\theoremstyle{plain}
\newtheorem{proposition}{Proposition}[section]
\theoremstyle{remark}
\AtBeginDocument{\renewcommand*{\proofname}{Proof}}
\newtheorem*{remark}{Remark}
\newtheorem*{solution}{Solution}
\makeatother
\makeatletter
\@ifpackageloaded{caption}{}{\usepackage{caption}}
\@ifpackageloaded{subcaption}{}{\usepackage{subcaption}}
\makeatother
\makeatletter
\@ifpackageloaded{tcolorbox}{}{\usepackage[many]{tcolorbox}}
\makeatother
\makeatletter
\@ifundefined{shadecolor}{\definecolor{shadecolor}{rgb}{.97, .97, .97}}
\makeatother
\makeatletter
\makeatother
\ifLuaTeX
  \usepackage{selnolig}  % disable illegal ligatures
\fi
\usepackage[citestyle = authoryear]{biblatex}
\addbibresource{../references.bib}
\IfFileExists{bookmark.sty}{\usepackage{bookmark}}{\usepackage{hyperref}}
\IfFileExists{xurl.sty}{\usepackage{xurl}}{} % add URL line breaks if available
\urlstyle{same} % disable monospaced font for URLs
\hypersetup{
  pdftitle={Galois Connections},
  pdfauthor={David A. Smith},
  pdfkeywords={galois connections, galois connection, ordered set},
  hidelinks,
  pdfcreator={LaTeX via pandoc}}

\title{Galois Connections}
\usepackage{etoolbox}
\makeatletter
\providecommand{\subtitle}[1]{% add subtitle to \maketitle
  \apptocmd{\@title}{\par {\large #1 \par}}{}{}
}
\makeatother
\subtitle{A Complete Introduction}
\author{David A. Smith}
\date{Thursday, February 9, 2023}

\begin{document}
\maketitle
\ifdefined\Shaded\renewenvironment{Shaded}{\begin{tcolorbox}[enhanced, frame hidden, borderline west={3pt}{0pt}{shadecolor}, sharp corners, boxrule=0pt, interior hidden, breakable]}{\end{tcolorbox}}\fi

\renewcommand*\contentsname{Table of contents}
{
\setcounter{tocdepth}{3}
\tableofcontents
}
\begin{example}[]\protect\hypertarget{exm-tautology-2}{}\label{exm-tautology-2}

Show that the statement \begin{equation}
\label{scsex}
(p\land \neg q)\rightarrow [(\neg p\lor \neg q)\rightarrow (p\land \neg q)]
\end{equation} is a tautology.

\end{example}

\hypertarget{introduction}{%
\section{Introduction}\label{introduction}}

\hypertarget{what-are-galois-connections}{%
\section{What are Galois
Connections?}\label{what-are-galois-connections}}

Let \((X,\preceq)\) and \((Y,\leqslant)\) be partially ordered sets.

\begin{definition}[]\protect\hypertarget{def-galois-connection}{}\label{def-galois-connection}

If \(f_*:X\to Y\) and \(f^*:Y\to X\) are functions such that
\begin{equation}
\label{gc}
f_*(x)\leqslant y \Longleftrightarrow x\preceq f^*(y)
\end{equation} for all \(x\in X\) and all \(y\in Y\), then
\((f_*, f^*)\) is called a \textbf{Galois connection} between
\((X,\preceq)\) and \((Y,\leqslant)\).

\end{definition}

There are several definitions of Galois connections in the literature;
however they are all order-isomorphic to the definition above.

\begin{proposition}[]\protect\hypertarget{prp-galois-connection-if-and-only-if}{}\label{prp-galois-connection-if-and-only-if}

Let \(f_*:X\to Y\) and \(f^*:Y\to X\) be functions. Then \((f_*, f^*)\)
is a Galois connection if and only if

\begin{enumerate}
\def\labelenumi{\arabic{enumi}.}
\tightlist
\item
  both \(f_*\) and \(f^*\) are monotone,
\item
  \(x\preceq (f^*\circ f_*)(x)\) for all \(x\in X\), and
\item
  \((f_* \circ f^*)(y)\leqslant y\) for all \(y\in Y\).
\end{enumerate}

\end{proposition}

\begin{proof}

Suppose \((f_*, f^*)\) is a Galois connection. By \eqref{gc} it follows
\[
f_*(x)\leqslant f_*(x) \Longleftrightarrow x\preceq (f^*\circ f_*)(x). 
\] Since \(\leqslant\) is reflexive, \(x\preceq (f^*\circ f_*)(x)\)
follows immediately. Similarly, by \eqref{gc} it follows \[
(f_*\circ f^*)(y)\leqslant y \Longleftrightarrow f^*(y)\preceq f^*(y)
\] proving that \eqref{gc3} also holds. Assume \(x_1\preceq x_2\). By
\eqref{gc2} we have \(x_2\preceq (f^*\circ f_*)(x_2)\). By \eqref{gc} it
follows \(f_*(x_1)\leqslant f_*(x_2)\) and so \(f_*\) is monotone.
Assume \(y_1\leqslant y_2\). By \eqref{gc3} we have
\((f_*\circ f^*)(y_1)\leqslant y_1\). By \eqref{gc} it follows
\(f^*(y_1)\preceq f^*(y_2)\) and so \(f^*\) is also monotone.

Conversely, assume \eqref{gc1}, \eqref{gc2}, and \eqref{gc3} all hold.
Assume \(f_*(x)\leqslant y\). By \eqref{gc1} and \eqref{gc2}, it follows
\((f^*\circ f_*)(x)\preceq f^*(y)\) and \(x\preceq (f^*\circ f_*)(x)\),
respectively. By transitivity of \(\preceq\), we have
\(x\preceq f^*(y)\) as needed. Assume \(x\preceq f^*(y)\). By
\eqref{gc1} and \eqref{gc3}, it follows
\(f_*(x)\leqslant (f_*\circ f^*)(y)\) and
\((f_*\circ f^*)(y)\leqslant y\). By transitivity of \(\leqslant\), we
have \(f_*(x)\leqslant y\) as needed. Therefore, \eqref{gc} holds and so
\((f_*, f^*)\) is a Galois connection.

\end{proof}

\begin{proposition}[]\protect\hypertarget{prp-galois-connections-uniqueness}{}\label{prp-galois-connections-uniqueness}

If \((f_*,g)\) and \((f_*, h)\) are Galois connections between
\((X,\preceq)\) and \((Y,\leqslant)\), then \(g=h\). Likewise, if
\((g, f^*)\) and \((h,f^*)\) are Galois connections between
\((X,\preceq)\) and \((Y,\leqslant)\), then \(g=h\).

\end{proposition}

\begin{proof}

Let \(f_*:X \to Y\) and \(g:Y\to X\) be a Galois connection. Also let
\(f_*\) and \(h:Y\to X\) be a Galois connection. By \eqref{gc} we have
the following\\
\begin{equation}
\label{ugc1}
f_*(x)\leqslant y \Longleftrightarrow x\preceq g(y)
\end{equation} \begin{equation}
\label{ugc2}
f_*(x)\leqslant y \Longleftrightarrow x\preceq h(y)
\end{equation} By \eqref{ugc1} we have
\((f_*\circ h)(y)\leqslant y\Longleftrightarrow h(y)\preceq g(y)\).
Notice \((f_*\circ h)(y)\leqslant y\) holds by \eqref{gcch}.\eqref{gc3};
and thus \(h(y)\preceq g(y)\). By \eqref{ugc2} we have
\((f_*\circ g)(y)\leqslant y \Longleftrightarrow g(y)\preceq h(y)\).
Notice \((f_*\circ g)(y)\leqslant y\) holds by \eqref{gcch}.\eqref{gc3};
and thus \(h(y)\preceq g(y)\). Since \(\preceq\) is antisymmetric, it
follows \(g(y)=h(y)\) for arbitrary \(y\). The second statement is the
dual of the first and follows just as easily using
\eqref{gcch}.\eqref{gc2}.

\end{proof}

If \((f,g)\) and \((f,h)\) are Galois connections between \((P,\le_P)\)
and \((Q,\le_Q)\), then \(g=h\). To see this, observe that
\(p\le_P g(q)\) iff \(f(p)\le_Q q\) iff \(p \le_P h(q)\), for any
\(p\in P\) and \(q\in Q\). In particular, setting \(p=g(q)\), we get
\(g(q)\le_P h(q)\) since \(g(q)\le_P g(q)\). Similarly,
\(h(q)\le_P g(q)\), and therefore \(g=h\). By a similarly argument, if
\((g,f)\) and \((h,f)\) are Galois connections between \((P,\le_P)\) and
\((Q,\le_Q)\), then \(g=h\). Because of this uniqueness property, in a
Galois connection \(f=(f^*,f_*)\), \(f^*\) is called \textbf{the upper
adjoint} of \(f_*\) and \(f_*\) \textbf{the lower adjoint} of \(f^*\).

\begin{proposition}[]\protect\hypertarget{prp-maximum-minimum}{}\label{prp-maximum-minimum}

If \((f_*, f^*)\) is a Galois connection between \((X,\preceq)\) and
\((Y,\leqslant)\), then

\begin{enumerate}
\def\labelenumi{\arabic{enumi}.}
\tightlist
\item
  \(f^*(y)=\textrm{the maximum of } \{x\in X : f_*(x)\leqslant y\}\) and
\item
  \(f_*(x)=\textrm{the minimum of } \{y\in Y : x\preceq f^*(y)\}\).
\end{enumerate}

\end{proposition}

\begin{proof}

Let \(M=\{x\in X: f_*(x)\leqslant y\}\). By \eqref{gcch}.\eqref{gc3} we
have \(f^*(y)\in M\). Let \(x\in M\). Then \(f_*(x)\leqslant y\) and
since \(f^*\) is monotone, it follows
\((f^*\circ f_*)(x)\preceq f^*(y)\). By \eqref{gcch}.\eqref{gc2}, we
have \(x\preceq (f^*\circ f_*)(x)\). By transitivity of \(\preceq\), we
have \(x\preceq f^*(y)\) and thus \(f^*(y)\) is the maximum of \(M\).
For the second statement, let \(N=\{y\in Y : x\preceq f^*(y)\}\). By
\eqref{gcch}.\eqref{gc2} we have \(f_*(x)\in N\). Let \(y\in N\). Then
\(x\preceq f^*(y)\) and since \(f_*\) is monotone, it follows
\(f_*(x)\leqslant (f_*\circ f^*)(y)\). By \eqref{gcch}.\eqref{gc3}, we
have \((f_*\circ f^*)(y)\leqslant y\) and so by transitivity, it follows
\(f_*(x)\leqslant y\). Thus \(f_*(x)\) is the minimum of \(N\).

\end{proof}

\begin{proposition}[]\protect\hypertarget{prp-}{}\label{prp-}

If \((f_*, f^*)\) is a Galois connection between \((X,\preceq)\) and
\((Y,\leqslant)\), then

\begin{enumerate}
\def\labelenumi{\arabic{enumi}.}
\tightlist
\item
  \(f_*\circ f^* \circ f_*=f_*\), \(f^*\circ f_* \circ f^*=f^*\),
\item
  \(x\in f^*(Y)\) if and only if \(x\) is a fixed point of
  \(f^*\circ f_*\),
\item
  \(y\in f_*(X)\) if and only if \(y\) is a fixed point of
  \(f_*\circ f^*\),
\item
  \(f^*(Y)=(f^*\circ f_*)(X)\), and \(f_*(X)=(f_*\circ f^*)(Y)\).
\end{enumerate}

\end{proposition}

\begin{proof}

\eqref{gcprop1}: Using \eqref{gcch}, we have
\(f_*(x)\leqslant (f_*\circ f^*\circ f_*)(x)\). By \eqref{gc} with
\(x:=(f^*\circ f_*)(x)\) and \(y:=f_*(x)\) it follows
\((f_*\circ f^* \circ f_*)(x)\leqslant f_*(x)\) using that \(\preceq\)
is reflexive.\\
Since \(\leqslant\) is antisymmetric, it follows
\(f_*(x)=(f_*\circ f^*\circ f_*)(x)\) for arbitrary \(x\), thus proving
\eqref{gcprop1} holds. \eqref{gcprop3}: By definition, \(x\in f^*(Y)\)
is equivalent to \(f^*(y)=x\) for some \(y\in Y\). Then \[
(f^*\circ f_*)(x)=(f^*\circ f_* \circ f^*)(y)=f^*(y)=x
\]\\
follows by \eqref{gcprop1}. \eqref{gcprop5}: It follows by
\eqref{gcprop3} that \(f^*(Y)\subseteq (f^*\circ f_*)(X)\). Conversely,
let \(x\in (f^*\circ f_*)(X)\). Then \(x=f^*(y)\) for some
\(y\in f_*(X)\subseteq Y\). By definition, \(x\in f^*(Y)\) and so
\((f^*\circ f_*)(X)\subseteq f^*(Y)\).

\end{proof}

\begin{proposition}[]\protect\hypertarget{prp-galois-connection-property}{}\label{prp-galois-connection-property}

If \((f_*, f^*)\) is a Galois connection between \((X,\preceq)\) and
\((Y,\leqslant)\), then

\begin{enumerate}
\def\labelenumi{\arabic{enumi}.}
\tightlist
\item
  \(x \preceq f^*(y) \Longleftrightarrow f_*(x)\leqslant (f_*\circ f^*)(y) \Longleftrightarrow f_*(x)\leqslant y \Longleftrightarrow (f^*\circ f_*)(x)\preceq f^*(y),\)
\item
  \(f^*(x)\preceq f^*(y)\Longleftrightarrow (f_*\circ f^*)(x)\leqslant (f_*\circ f^*)(y) \Longleftrightarrow (f_*\circ f^*)(x)\leqslant y\),
  and
\item
  \(f_*(x)\leqslant f_*(y) \Longleftrightarrow (f^*\circ f_*)(x)\preceq (f^*\circ f_*)(y) \Longleftrightarrow x\preceq (f^*\circ f_*)(x)\),
\item
  \(f^*(x)=f^*(y) \Longleftrightarrow (f_*\circ f^*)(x)=(f_*\circ f^*)(y)\),
  and
\item
  \(f_*(x)=f_*(y) \Longleftrightarrow (f^*\circ f_*)(x)=(f^*\circ f_*)(y)\).
\end{enumerate}

\end{proposition}

\begin{proof}

For the first statement we have \begin{align*}
x \preceq f^*(y) 
& \implies f_*(x)\leqslant (f_*\circ f^*)(y) 
\implies f_*(x)\leqslant y \\
& \implies (f^*\circ f_*)(x) \preceq f^*(y) 
\implies x\preceq f^*(y)  
\end{align*} For the third statement we have \begin{align*}
f^*(x) \preceq f^*(y) 
& \implies (f_*\circ f_*)(x)\leqslant (f_*\circ f^*)(y) 
 \implies (f_*\circ f_*)(x)\leqslant y 
\end{align*} For the fourth statement we have \begin{align*}
f^*(x)=f^*(y) 
& \Longleftrightarrow f^*(x)\preceq f^*(y) \land f^*(y) \preceq f^*(x) \\
& \Longleftrightarrow (f_*\circ f^*)(x)\leqslant (f_*\circ f^*)(y) \land  (f_*\circ f^*)(y)\leqslant (f_*\circ f^*)(x) \\
& \Longleftrightarrow (f_*\circ f^*)(x)=(f_*\circ f^*)(y) 
\end{align*} The remaining statements are the dual and easily proved.

\end{proof}

By an \textbf{order isomorphism} from an partially ordered set \(X\) to
another partially ordered set \(Y\) we shall mean an isotone bijection
\(f:X\to Y\) whose inverse \(f^{-1}: Y\to X\) is also an isotone.

\begin{proposition}[]\protect\hypertarget{prp-isomorphic-ordered-sets}{}\label{prp-isomorphic-ordered-sets}

Partially ordered sets \((X,\preceq)\) and \((Y,\leqslant)\) are
isomorphic if and only if there is a surjective mapping \(f:X\to Y\)
such that \[
x\preceq y \Longleftrightarrow f(x)\leqslant f(y).
\]

\end{proposition}

\begin{proof}

The necessity is clear. Suppose conversely that such a surjective
mapping \(f\) exists. Then \(f\) is also injective; for if \(f(x)=f(y)\)
then from \(f(x)\leqslant f(y)\) we obtain \(x\preceq y\) and from
\(f(y)\leqslant f(x)\) we obtain \(y\preceq x\), so that \(x=y\). Hence
\(f\) is a bijection. Clearly, \(f\) is isotone; and so also is
\(f^{-1}\), since \(x\preceq y\) can be written
\(f(f^{-1}))(x)\leqslant f(f^{-1})(y)\) which gives
\(f^{-1}(x)\preceq f^{-1}(y)\).

\end{proof}

\begin{proposition}[]\protect\hypertarget{prp-galois-connection-order-isomorphic}{}\label{prp-galois-connection-order-isomorphic}

\label{calcprop1} If \((f_*, f^*)\) is a Galois connection between
\((X,\preceq)\) and \((Y,\leqslant)\), then \(f^*(Y)\) and \(f_*(X)\)
are order-isomorphic.

\end{proposition}

\begin{proof}

This follows immediately from \eqref{gcch} and \eqref{posetiso}.

\end{proof}

\begin{definition}[]\protect\hypertarget{def-closure-function}{}\label{def-closure-function}

A function \(f\) on \(X\) is called a \textbf{(co-)closure function} if

\begin{enumerate}
\def\labelenumi{\arabic{enumi}.}
\tightlist
\item
  \(f\) is extensive, \(\forall x\in X, x\preceq f(x)\),
  (\(\forall x\in X, f(x)\preceq x\)),
\item
  \(f\) is monotone, \(x_1\preceq x_2\implies f(x_1)\preceq f(x_2)\),
  for all \(x_1, x_2\in X\), and
\item
  \(f\) is idempotent, \(f(f(x))=f(x)\) for all \(x\in X\).
\end{enumerate}

\end{definition}

\begin{proposition}[]\protect\hypertarget{prp-galois-connection-closure-function}{}\label{prp-galois-connection-closure-function}

If \((f_*, f^*)\) is a Galois connection between \((X,\preceq)\) and
\((Y,\leqslant)\), then \(f^*\circ f_*\) is a closure function for \(X\)
and \(f_*\circ f^*\) is a co-closure function for \(Y\).

\end{proposition}

\begin{proof}

Let \(x\in X\). Then \(x\preceq (f^*\circ f_*)(x)\) follows by
\eqref{gcch}.\eqref{gc2}. Since both \(f^*\) and \(f_*\) are monotone we
have,
\(x_1\preceq x_2 \implies (f^*\circ f_*)(x_1)\preceq (f^*\circ f_*)(x_2)\).
Thus \(f^*\circ f_*\) is also monotone. By associativity of functions
and \eqref{gcprop}.\eqref{gcprop1} we have \[
(f^*\circ f_*)\circ (f^*\circ f_*)=f^*\circ (f_*\circ f^*\circ f_*)=f^*\circ f_*
\]\\
as needed. The dual statement is proved just as easily.

\end{proof}

By \eqref{gcprop}, the closed elements of \(f^*\circ f_*\) and
\(f_*\circ f^*\) are precisely the elements that are an image of some
element under \(f^*\), respectively \(f_*\).

There is a one-to-one correspondence between Galois connections and
(co-)closure functions.

\begin{proposition}[]\protect\hypertarget{prp-closure-galois-connection}{}\label{prp-closure-galois-connection}

If \(f\) is a closure (respectively co-closure) function, then there is
a Galois connection \((f_*,f^*)\) such that \(f=f^*\circ f_*\)
(respectively \(f=f_*\circ f^*\)).

\end{proposition}

\begin{proof}

Let \(f:X\to X\) be a closure over \((X,\preceq)\). Let \(\overline{X}\)
be the set of closed elements of \(f\) that is \(f(X)=\overline{X}\). We
will construct a Galois connection between \(\overline{X}\) and \(X\)
using \eqref{gcch}. Let \(f_*=f\), that is \(f_*:X\to \overline{X}\)
defined by \(f_*(x)=f(x)\) for all \(x\in X\). Let
\(f^*:\overline{X}\to X\) be the inclusion mapping, that is \(f^*(x)=x\)
for all \(x\in \overline{X}\). Notice \(f_*\circ f^*\) is the identity
on \(\overline{X}\) and \(f^*\circ f_*=f\).

\begin{enumerate}
\item
Notice $f_*$ is monotone since $f$ is monotone and that $f^*$ is monotone since the identity is monotone. 
\item
Let $x\in X$. 
Since $f$ is extensive, we have $x\preceq f(x)$. 
Thus it follows,
\[
x\preceq f(x)=(f^*\circ f)(x)=(f^*\circ f_*)(x).
\]
\item
Let $y\in \overline{X}$. 
There exists $x\in X$ such that $y=f(x)$. 
Since $f$ is idempotent we have
\[
f(y)=(f\circ f)(y)\preceq y
\]
for all $y\in \overline{X}$ as needed.
\end{enumerate}

Therefore, \((f_*,f^*)=(f,f^*)\) where \(f^*:\overline{X}\to X\) is the
inclusion mapping is a Galois connection between \((X,\prec0\) and
\((\overline{X},\preceq)\).

\end{proof}

\begin{proposition}[]\protect\hypertarget{prp-}{}\label{prp-}

\label{gcrel} Let \(R\) be a relation between \(X\) and \(Y\). Let
\begin{equation}
\label{gcrel1}
f_R(A)=\{b\in Y:\forall a (a\in A\implies (a,b)\in R) \} \text{ and }
\end{equation} \begin{equation}
\label{gcrel2}
f^R(B)=\{a\in X:\forall b (b\in B\implies (a,b)\in R) \}.
\end{equation} Then \((f_R, f^R)\) is a Galois connection between
\((P(X),\subseteq)\) and \((P(Y),\supseteq)\)

\end{proposition}

\begin{proof}

Clearly, \(f_R:P(X)\to P(Y)\) and \(f^R:P(Y)\to P(X)\) are functions. By
\eqref{gc} we must show \begin{equation}
\label{gcrel}
f_R(A)\supseteq B \Longleftrightarrow A\subseteq f^R(B)
\end{equation} for all \(A\in P(X)\) and all \(B\in P(Y)\). Assume
\(B\subseteq f_R(A)\). We will show \(A\subseteq f^R(B)\). Let
\(x\in A\). If \(y\in B\), then \(y\in f_R(A)\). Then, by
\eqref{gcrel1}, it follows \((x,y)\in R\). So we have shown,
\(y\in B\implies (x,y)\in R\) as needed to show \(x\in f^R(B)\).
Conversely, assume \(A\subseteq f^R(B)\). We will show
\(B\subseteq f_R(A)\). Let \(y\in B\). If \(x\in A\), then
\(x\in f^R(B)\). Then, by \eqref{gcrel2}, it follows \((x,y)\in R\). So
we have shown, \(x\in A\implies (x,y)\in R\) as needed to show
\(y\in f_R(A)\). Therefore, \eqref{gcrel} holds.

\end{proof}


\printbibliography



\thispagestyle{empty}


\end{document}
