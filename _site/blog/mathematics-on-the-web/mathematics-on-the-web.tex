% Options for packages loaded elsewhere
\PassOptionsToPackage{unicode}{hyperref}
\PassOptionsToPackage{hyphens}{url}
%
\documentclass[
  twoside,
  12pt,
  letterpaper,
  fleqn]{article}

\usepackage{amsmath,amssymb}
\usepackage{lmodern}
\usepackage{iftex}
\ifPDFTeX
  \usepackage[T1]{fontenc}
  \usepackage[utf8]{inputenc}
  \usepackage{textcomp} % provide euro and other symbols
\else % if luatex or xetex
  \usepackage{unicode-math}
  \defaultfontfeatures{Scale=MatchLowercase}
  \defaultfontfeatures[\rmfamily]{Ligatures=TeX,Scale=1}
\fi
% Use upquote if available, for straight quotes in verbatim environments
\IfFileExists{upquote.sty}{\usepackage{upquote}}{}
\IfFileExists{microtype.sty}{% use microtype if available
  \usepackage[]{microtype}
  \UseMicrotypeSet[protrusion]{basicmath} % disable protrusion for tt fonts
}{}
\makeatletter
\@ifundefined{KOMAClassName}{% if non-KOMA class
  \IfFileExists{parskip.sty}{%
    \usepackage{parskip}
  }{% else
    \setlength{\parindent}{0pt}
    \setlength{\parskip}{6pt plus 2pt minus 1pt}}
}{% if KOMA class
  \KOMAoptions{parskip=half}}
\makeatother
\usepackage{xcolor}
\setlength{\emergencystretch}{3em} % prevent overfull lines
\setcounter{secnumdepth}{-\maxdimen} % remove section numbering
% Make \paragraph and \subparagraph free-standing
\ifx\paragraph\undefined\else
  \let\oldparagraph\paragraph
  \renewcommand{\paragraph}[1]{\oldparagraph{#1}\mbox{}}
\fi
\ifx\subparagraph\undefined\else
  \let\oldsubparagraph\subparagraph
  \renewcommand{\subparagraph}[1]{\oldsubparagraph{#1}\mbox{}}
\fi


\providecommand{\tightlist}{%
  \setlength{\itemsep}{0pt}\setlength{\parskip}{0pt}}\usepackage{longtable,booktabs,array}
\usepackage{calc} % for calculating minipage widths
% Correct order of tables after \paragraph or \subparagraph
\usepackage{etoolbox}
\makeatletter
\patchcmd\longtable{\par}{\if@noskipsec\mbox{}\fi\par}{}{}
\makeatother
% Allow footnotes in longtable head/foot
\IfFileExists{footnotehyper.sty}{\usepackage{footnotehyper}}{\usepackage{footnote}}
\makesavenoteenv{longtable}
\usepackage{graphicx}
\makeatletter
\def\maxwidth{\ifdim\Gin@nat@width>\linewidth\linewidth\else\Gin@nat@width\fi}
\def\maxheight{\ifdim\Gin@nat@height>\textheight\textheight\else\Gin@nat@height\fi}
\makeatother
% Scale images if necessary, so that they will not overflow the page
% margins by default, and it is still possible to overwrite the defaults
% using explicit options in \includegraphics[width, height, ...]{}
\setkeys{Gin}{width=\maxwidth,height=\maxheight,keepaspectratio}
% Set default figure placement to htbp
\makeatletter
\def\fps@figure{htbp}
\makeatother


\usepackage{geometry}

\geometry{reset, letterpaper, height=9in, width=6in, hmarginratio=1:1, vmarginratio=1:1, marginparsep=0pt, marginparwidth=0pt, headheight=15pt}

\usepackage{lipsum}

\def\dklogo{\raisebox{-.2\height}{\includegraphics[width=12pt]{../../assets/direct-knowledge-logo-book.png}}\footnotesize{\ Direct Knowledge}}
\usepackage{fancyhdr}
\pagestyle{fancy}
\fancyhead[LE]{\thepage}%
\fancyhead[RE]{\dklogo}%
\fancyhead[CE]{}%
\fancyfoot[LE]{}%
\fancyfoot[RE]{}%
\fancyfoot[CE]{}%
\fancyhead[LO]{D. A. Smith}%
\fancyhead[RO]{\thepage}%
\fancyhead[CO]{}%
\fancyfoot[LO]{}%
\fancyfoot[RO]{}%
\fancyfoot[CO]{}%


\renewcommand{\headrulewidth}{0.4pt}
\renewcommand{\footrulewidth}{0pt}

\usepackage{setspace}
\linespread{1.25}

\usepackage{amssymb}
\usepackage{amsmath}
\usepackage{amsthm}
\usepackage{graphicx} 


\usepackage[most]{tcolorbox} 
\definecolor{block-gray}{gray}{0.97}
\newtcolorbox{zitat}[1][]{%
    colback=block-gray,
    grow to right by=-10mm,
    grow to left by=-10mm, 
    boxrule=0pt,
    boxsep=0pt,
    breakable,
    enhanced jigsaw,
    borderline west={2pt}{0pt}{gray},
    colbacktitle={block-gray},
    coltitle={black},
    fonttitle={\large\bfseries},
    attach title to upper={},
    #1,
}
\renewcommand{\quote}{\zitat}
\renewcommand{\endquote}{\endzitat}

\def\banner{ {\rule{\linewidth}{0.2pt}}
    \begin{minipage}[c]{14px}\includegraphics[width=14pt]{../../assets/direct-knowledge-logo-book.png}\end{minipage}
    \begin{minipage}[c]{150px}Direct Knowledge | Blog \end{minipage}
}

\usepackage{titling}
\setlength{\droptitle}{-8ex}
\pretitle{\vspace{-20pt}\banner\vspace{10pt}\begin{flushleft}\huge\bfseries}
\posttitle{\par\end{flushleft}}
\preauthor{\begin{flushleft}\Large}
\postauthor{,
    \small{Direct Knowledge, USA}
    \footnote{email: david@directknowledge.com}
    \footnote{\copyright \, 2023 \ David A. Smith}
    \footnote{With authorization, you can freely share and reproduce portions of this work for educational or personal use. Please note that distributing any portion of it in print form requires further permission from its original authors, as does posting online to public servers or mailing lists without prior consent.}
    \end{flushleft}
    }
\predate{\begin{flushleft}}
\postdate{\end{flushleft}}

\renewenvironment{abstract}
{\par\noindent\textbf{\abstractname.}\ \ignorespaces \itshape}
{\par\medskip}


\makeatletter
\makeatother
\makeatletter
\makeatother
\makeatletter
\@ifpackageloaded{caption}{}{\usepackage{caption}}
\AtBeginDocument{%
\ifdefined\contentsname
  \renewcommand*\contentsname{Table of contents}
\else
  \newcommand\contentsname{Table of contents}
\fi
\ifdefined\listfigurename
  \renewcommand*\listfigurename{List of Figures}
\else
  \newcommand\listfigurename{List of Figures}
\fi
\ifdefined\listtablename
  \renewcommand*\listtablename{List of Tables}
\else
  \newcommand\listtablename{List of Tables}
\fi
\ifdefined\figurename
  \renewcommand*\figurename{Figure}
\else
  \newcommand\figurename{Figure}
\fi
\ifdefined\tablename
  \renewcommand*\tablename{Table}
\else
  \newcommand\tablename{Table}
\fi
}
\@ifpackageloaded{float}{}{\usepackage{float}}
\floatstyle{ruled}
\@ifundefined{c@chapter}{\newfloat{codelisting}{h}{lop}}{\newfloat{codelisting}{h}{lop}[chapter]}
\floatname{codelisting}{Listing}
\newcommand*\listoflistings{\listof{codelisting}{List of Listings}}
\makeatother
\makeatletter
\@ifpackageloaded{caption}{}{\usepackage{caption}}
\@ifpackageloaded{subcaption}{}{\usepackage{subcaption}}
\makeatother
\makeatletter
\@ifpackageloaded{tcolorbox}{}{\usepackage[many]{tcolorbox}}
\makeatother
\makeatletter
\@ifundefined{shadecolor}{\definecolor{shadecolor}{rgb}{.97, .97, .97}}
\makeatother
\makeatletter
\makeatother
\ifLuaTeX
  \usepackage{selnolig}  % disable illegal ligatures
\fi
\usepackage[citestyle = authoryear]{biblatex}
\addbibresource{../references.bib}
\IfFileExists{bookmark.sty}{\usepackage{bookmark}}{\usepackage{hyperref}}
\IfFileExists{xurl.sty}{\usepackage{xurl}}{} % add URL line breaks if available
\urlstyle{same} % disable monospaced font for URLs
\hypersetup{
  pdftitle={Mathematics on the Web},
  pdfauthor={David A. Smith},
  pdfkeywords={mathematics, web, publishing, acknowledgements},
  hidelinks,
  pdfcreator={LaTeX via pandoc}}

\title{Mathematics on the Web}
\usepackage{etoolbox}
\makeatletter
\providecommand{\subtitle}[1]{% add subtitle to \maketitle
  \apptocmd{\@title}{\par {\large #1 \par}}{}{}
}
\makeatother
\subtitle{Acknowledgements}
\author{David A. Smith}
\date{Friday, January 27, 2023}

\begin{document}
\maketitle
\begin{abstract}
How do I do it? How do I publish mathematics books on the web? How do I
publish interactive, programmable, and full-media mathematical
experiences on the web? Find out by reading the acknowledgements listed
below.
\end{abstract}
\ifdefined\Shaded\renewenvironment{Shaded}{\begin{tcolorbox}[frame hidden, sharp corners, borderline west={3pt}{0pt}{shadecolor}, boxrule=0pt, interior hidden, enhanced, breakable]}{\end{tcolorbox}}\fi

\hypertarget{introduction}{%
\subsection{Introduction}\label{introduction}}

This article does not explain why it is arduous to publish mathematics
on the web, but rather briefly outlines the solution to the problem that
I have implemented.\footnote{If you are interested in the reasons why
  it's so difficult, we'll that has already been written by Brian Hayes:
  \href{https://www.americanscientist.org/article/writing-math-on-the-web}{Writing
  Math on the Web} (see \textcite{brainhayes}).}

It is impossible to overstate the importance of those who helped bring
\protect\hyperlink{mathematics-books-on-the-web}{these books into their
completed form}. I am deeply grateful for their generous contributions
and would like to express my sincere appreciation here.

\hypertarget{mathjax}{%
\subsection{MathJax}\label{mathjax}}

MathJax is a powerful open source typesetting platform that enables
users to generate beautiful, high-quality mathematical and scientific
content on the web. With MathJax, complex equations are no longer an
obstacle in educational or professional communication (see
\textcite{cervone2012mathjax}, \textcite{Davide}) - they can be
displayed accurately with ease. MathJax deserves a tremendous amount of
gratitude for \href{https://www.mathjax.org}{enabling mathematical
equations and formulas to be written in HTML documents}.

\begin{quote}
I am truly thankful for the use of this incredible tool.
\end{quote}

\hypertarget{python}{%
\subsection{Python}\label{python}}

The Python programming language has become
\href{https://pypl.github.io/PYPL.html}{increasingly sought-after} for
its versatile capabilities and ease of use. Its prominence in a wide
range of industries has seen continued growth (see
\textcite{Millman2011}), making it one of the most popular languages on
the market today.

\begin{quote}
I'd like to offer a hearty salute and our sincerest gratitude to all
those who have contributed their time, effort, expertise, and knowledge
in \href{https://www.python.org}{developing python}. Your contributions
are invaluable; without them these books would not have been possible.
\end{quote}

\hypertarget{pyscript}{%
\subsection{PyScript}\label{pyscript}}

PyScript enables users of all backgrounds to create powerful
Python-based web applications with ease.
\href{https://pyscript.net}{PyScript revolutionizes the way users can
create engaging}, dynamic Python apps with a hybrid of HTML and its
versatile language. By harnessing this cutting-edge technology (see
\textcite{Arepalli2022}), learners are granted unprecedented access to
powerful tools previously inaccessible in browser environments.

\begin{quote}
I am deeply grateful to the many pioneering minds that have advanced
pyscript into a powerful tool for making learning easier.
\end{quote}

\hypertarget{quarto}{%
\subsection{Quarto}\label{quarto}}

\href{https://quarto.org}{Quarto} is an open-source system designed to
revolutionize scientific and technical publishing. The platform allows
authors to expertly craft dynamic content with Python, R, Julia or
Observable - in plain text markdown or Jupyter notebooks that can be
published as high quality articles, reports, presentations, websites,
blogs, and books in html, pdf, and epub formats. With the help of
scientific markdown you'll have access to equations, citations,
crossrefs, figure, panels, callouts, plus advanced layout options
(e.g.~see \textcite{aust}).

\begin{quote}
Quarto has been instrumental in enabling me to realize my ambitions and
stretch my creative ideas beyond what I ever imagined was possible.
Their service created a platform for success, fuelling creativity until
there were no boundaries left standing - I am immensely thankful.
\end{quote}

\hypertarget{pandoc}{%
\subsection{Pandoc}\label{pandoc}}

\href{https://pandoc.org}{Pandoc} is a powerful tool for the modern
writer, transforming documents into all sorts of formats with ease. It's
free-software too, forming an essential part in many publishing
workflows (see \textcite{krewinkel2017formatting}) due to its breadth
and simplicity. In an age of digital transformation, open source
technology has presented innovative yet powerful tools to create
beautiful documents with ease and confidence (see
\textcite{mailund2019introducing}).

\begin{quote}
Pandoc stands out as a shining example -- expressing our sincere
appreciation for this must-have resource.
\end{quote}

\hypertarget{visual-studio-code}{%
\subsection{Visual Studio Code}\label{visual-studio-code}}

\href{https://code.visualstudio.com}{Visual Studio Code} is a powerful,
cross-platform source-code editor created by Microsoft. It boasts
intuitive features like debugging support, syntax highlighting and code
refactoring - not to mention embedded Git - that make coding faster and
easier than ever before.

\begin{quote}
VS Code has been key in helping us create, build, and refine our work -
for that I am immensely grateful.
\end{quote}

\hypertarget{github}{%
\subsection{GitHub}\label{github}}

\href{https://github.com/about}{GitHub} offers a comprehensive platform
for software development and version control using Git, transforming the
process of crafting quality code. Through tools such as access control,
bug tracking, feature requests, and discussions
(\textcite{hata2022github}) developers are able to efficiently create
new projects.

\begin{quote}
GitHub is a revolutionary resource that has profoundly impacted the
world. I express my deep gratitude for its existence.
\end{quote}

\hypertarget{netlify}{%
\subsection{Netlify}\label{netlify}}

\href{https://www.netlify.com}{Netlify} empowers developers to create,
launch, and manage web applications with innovative cloud computing
services. Their platform streamlines the process of building robust
sites that can quickly scale or deploy serverless backends for dynamic
experiences.

\begin{quote}
I am incredibly thankful for using Netlify - a powerful platform that
helps streamline website development and deployment. I look forward to a
successful journey together.
\end{quote}

\hypertarget{revealjs}{%
\subsection{RevealJS}\label{revealjs}}

\href{https://revealjs.com}{RevealJS} is an open source HTML
presentation framework that allows you to create captivating
presentations without cost or hassle.

\begin{quote}
I offer my sincere gratitude to the Reveal.js community for providing us
with a great platform that allows us to craft engaging and interactive
stories. Their library has enabled us to bring forth innovative ideas in
an impactful way.
\end{quote}

\hypertarget{youtube}{%
\subsection{YouTube}\label{youtube}}

\href{https://about.youtube}{YouTube is an invaluable platform} that we
are proud to be part of.

\begin{quote}
I would like to express our admiration and appreciation for all the
viewers who support us with their viewership, allowing YouTube's vast
influence on culture to reach ever further.
\end{quote}

\hypertarget{direct-knowledge}{%
\subsection{Direct Knowledge}\label{direct-knowledge}}

So if you're new to any of the technologies listed above, you might
question how I bring all this together. In case so, here's a brief
explanation.

I use Visual Studio Code to write in Html, Css, Js, Markdown, Quarto
Markdown, Python, and TeX. After setting up a few configuration files
(one-time only), I issue the command ``quarto render'' to build the
static files. The website, books, blog(s), and slides are outputs from
this building process. Math is displayed on the webpages using MathJax,
Python runs in the browser using PyScript, and the pdfs are created
using an installation of LaTeX in the background via Pandoc. These files
are uploaded to a Github repository and then published using Netlify. I
then use the ReavalJS slideshows to make YouTube videos. I have found
that by using Visual Studio Code and Quarto together, this whole process
takes place very quickly and is without cost.

\begin{quote}
These books, and their presentations, wouldn't have been possible
without those special individuals who offered their insight and
guidance. I am profoundly grateful to them for supporting my work.
\end{quote}


\printbibliography



\thispagestyle{empty}


\end{document}
