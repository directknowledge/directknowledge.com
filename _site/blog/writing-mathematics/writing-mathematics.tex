% Options for packages loaded elsewhere
\PassOptionsToPackage{unicode}{hyperref}
\PassOptionsToPackage{hyphens}{url}
%
\documentclass[
  twoside,
  12pt,
  letterpaper,
  fleqn]{article}

\usepackage{amsmath,amssymb}
\usepackage{lmodern}
\usepackage{iftex}
\ifPDFTeX
  \usepackage[T1]{fontenc}
  \usepackage[utf8]{inputenc}
  \usepackage{textcomp} % provide euro and other symbols
\else % if luatex or xetex
  \usepackage{unicode-math}
  \defaultfontfeatures{Scale=MatchLowercase}
  \defaultfontfeatures[\rmfamily]{Ligatures=TeX,Scale=1}
\fi
% Use upquote if available, for straight quotes in verbatim environments
\IfFileExists{upquote.sty}{\usepackage{upquote}}{}
\IfFileExists{microtype.sty}{% use microtype if available
  \usepackage[]{microtype}
  \UseMicrotypeSet[protrusion]{basicmath} % disable protrusion for tt fonts
}{}
\makeatletter
\@ifundefined{KOMAClassName}{% if non-KOMA class
  \IfFileExists{parskip.sty}{%
    \usepackage{parskip}
  }{% else
    \setlength{\parindent}{0pt}
    \setlength{\parskip}{6pt plus 2pt minus 1pt}}
}{% if KOMA class
  \KOMAoptions{parskip=half}}
\makeatother
\usepackage{xcolor}
\setlength{\emergencystretch}{3em} % prevent overfull lines
\setcounter{secnumdepth}{-\maxdimen} % remove section numbering
% Make \paragraph and \subparagraph free-standing
\ifx\paragraph\undefined\else
  \let\oldparagraph\paragraph
  \renewcommand{\paragraph}[1]{\oldparagraph{#1}\mbox{}}
\fi
\ifx\subparagraph\undefined\else
  \let\oldsubparagraph\subparagraph
  \renewcommand{\subparagraph}[1]{\oldsubparagraph{#1}\mbox{}}
\fi


\providecommand{\tightlist}{%
  \setlength{\itemsep}{0pt}\setlength{\parskip}{0pt}}\usepackage{longtable,booktabs,array}
\usepackage{calc} % for calculating minipage widths
% Correct order of tables after \paragraph or \subparagraph
\usepackage{etoolbox}
\makeatletter
\patchcmd\longtable{\par}{\if@noskipsec\mbox{}\fi\par}{}{}
\makeatother
% Allow footnotes in longtable head/foot
\IfFileExists{footnotehyper.sty}{\usepackage{footnotehyper}}{\usepackage{footnote}}
\makesavenoteenv{longtable}
\usepackage{graphicx}
\makeatletter
\def\maxwidth{\ifdim\Gin@nat@width>\linewidth\linewidth\else\Gin@nat@width\fi}
\def\maxheight{\ifdim\Gin@nat@height>\textheight\textheight\else\Gin@nat@height\fi}
\makeatother
% Scale images if necessary, so that they will not overflow the page
% margins by default, and it is still possible to overwrite the defaults
% using explicit options in \includegraphics[width, height, ...]{}
\setkeys{Gin}{width=\maxwidth,height=\maxheight,keepaspectratio}
% Set default figure placement to htbp
\makeatletter
\def\fps@figure{htbp}
\makeatother


\usepackage{geometry}

\geometry{reset, letterpaper, height=9in, width=6in, hmarginratio=1:1, vmarginratio=1:1, marginparsep=0pt, marginparwidth=0pt, headheight=15pt}

\usepackage{lipsum}

\def\dklogo{\raisebox{-.2\height}{\includegraphics[width=12pt]{../../assets/direct-knowledge-logo-book.png}}\footnotesize{\ Direct Knowledge}}
\usepackage{fancyhdr}
\pagestyle{fancy}
\fancyhead[LE]{\thepage}%
\fancyhead[RE]{\dklogo}%
\fancyhead[CE]{}%
\fancyfoot[LE]{}%
\fancyfoot[RE]{}%
\fancyfoot[CE]{}%
\fancyhead[LO]{D. A. Smith}%
\fancyhead[RO]{\thepage}%
\fancyhead[CO]{}%
\fancyfoot[LO]{}%
\fancyfoot[RO]{}%
\fancyfoot[CO]{}%


\renewcommand{\headrulewidth}{0.4pt}
\renewcommand{\footrulewidth}{0pt}

\usepackage{setspace}
\linespread{1.25}

\usepackage{amssymb}
\usepackage{amsmath}
\usepackage{amsthm}
\usepackage{graphicx} 


\usepackage[most]{tcolorbox} 
\definecolor{block-gray}{gray}{0.97}
\newtcolorbox{zitat}[1][]{%
    colback=block-gray,
    grow to right by=-10mm,
    grow to left by=-10mm, 
    boxrule=0pt,
    boxsep=0pt,
    breakable,
    enhanced jigsaw,
    borderline west={2pt}{0pt}{gray},
    colbacktitle={block-gray},
    coltitle={black},
    fonttitle={\large\bfseries},
    attach title to upper={},
    #1,
}
\renewcommand{\quote}{\zitat}
\renewcommand{\endquote}{\endzitat}

\def\banner{ {\rule{\linewidth}{0.2pt}}
    \begin{minipage}[c]{14px}\includegraphics[width=14pt]{../../assets/direct-knowledge-logo-book.png}\end{minipage}
    \begin{minipage}[c]{150px}Direct Knowledge | Blog \end{minipage}
}

\usepackage{titling}
\setlength{\droptitle}{-8ex}
\pretitle{\vspace{-20pt}\banner\vspace{10pt}\begin{flushleft}\huge\bfseries}
\posttitle{\par\end{flushleft}}
\preauthor{\begin{flushleft}\Large}
\postauthor{,
    \small{Direct Knowledge, USA}
    \footnote{email: david@directknowledge.com}
    \footnote{\copyright \, 2023 \ David A. Smith}
    \footnote{With authorization, you can freely share and reproduce portions of this work for educational or personal use. Please note that distributing any portion of it in print form requires further permission from its original authors, as does posting online to public servers or mailing lists without prior consent.}
    \end{flushleft}
    }
\predate{\begin{flushleft}}
\postdate{\end{flushleft}}

\renewenvironment{abstract}
{\par\noindent\textbf{\abstractname.}\ \ignorespaces \itshape}
{\par\medskip}


\makeatletter
\makeatother
\makeatletter
\makeatother
\makeatletter
\@ifpackageloaded{caption}{}{\usepackage{caption}}
\AtBeginDocument{%
\ifdefined\contentsname
  \renewcommand*\contentsname{Table of contents}
\else
  \newcommand\contentsname{Table of contents}
\fi
\ifdefined\listfigurename
  \renewcommand*\listfigurename{List of Figures}
\else
  \newcommand\listfigurename{List of Figures}
\fi
\ifdefined\listtablename
  \renewcommand*\listtablename{List of Tables}
\else
  \newcommand\listtablename{List of Tables}
\fi
\ifdefined\figurename
  \renewcommand*\figurename{Figure}
\else
  \newcommand\figurename{Figure}
\fi
\ifdefined\tablename
  \renewcommand*\tablename{Table}
\else
  \newcommand\tablename{Table}
\fi
}
\@ifpackageloaded{float}{}{\usepackage{float}}
\floatstyle{ruled}
\@ifundefined{c@chapter}{\newfloat{codelisting}{h}{lop}}{\newfloat{codelisting}{h}{lop}[chapter]}
\floatname{codelisting}{Listing}
\newcommand*\listoflistings{\listof{codelisting}{List of Listings}}
\makeatother
\makeatletter
\@ifpackageloaded{caption}{}{\usepackage{caption}}
\@ifpackageloaded{subcaption}{}{\usepackage{subcaption}}
\makeatother
\makeatletter
\@ifpackageloaded{tcolorbox}{}{\usepackage[many]{tcolorbox}}
\makeatother
\makeatletter
\@ifundefined{shadecolor}{\definecolor{shadecolor}{rgb}{.97, .97, .97}}
\makeatother
\makeatletter
\makeatother
\ifLuaTeX
  \usepackage{selnolig}  % disable illegal ligatures
\fi
\usepackage[citestyle = authoryear]{biblatex}
\addbibresource{../references.bib}
\IfFileExists{bookmark.sty}{\usepackage{bookmark}}{\usepackage{hyperref}}
\IfFileExists{xurl.sty}{\usepackage{xurl}}{} % add URL line breaks if available
\urlstyle{same} % disable monospaced font for URLs
\hypersetup{
  pdftitle={Writing Mathematics},
  pdfauthor={David A. Smith},
  pdfkeywords={mathematics, writing, publishing},
  hidelinks,
  pdfcreator={LaTeX via pandoc}}

\title{Writing Mathematics}
\usepackage{etoolbox}
\makeatletter
\providecommand{\subtitle}[1]{% add subtitle to \maketitle
  \apptocmd{\@title}{\par {\large #1 \par}}{}{}
}
\makeatother
\subtitle{How I Write Mathematics}
\author{David A. Smith}
\date{Saturday, January 28, 2023}

\begin{document}
\maketitle
\begin{abstract}
Writing mathematics clearly, carefully, and accurately is a meaningful
skill for people of all ages to learn and understand. It requires
considerable focus, such as understanding concepts with specialized
language and notation. Writing mathematics can be difficult, but it is
worth pursuing because it helps people comprehend complex ideas and
prepares them for advancing their own.
\end{abstract}
\ifdefined\Shaded\renewenvironment{Shaded}{\begin{tcolorbox}[sharp corners, borderline west={3pt}{0pt}{shadecolor}, breakable, boxrule=0pt, enhanced, interior hidden, frame hidden]}{\end{tcolorbox}}\fi

One of the greatest challenges for students is learning to express
mathematics fluently and accurately. Trying to decipher complex formulas
or explaining mathematical ideas in writing can be a daunting task, even
for those with an advanced understanding of mathematics. However,
mastering this skill can result in tremendous rewards. Expressing
mathematical ideas effectively through written language provides us with
a greater understanding and appreciation of the subject, not only from
an academic perspective but from its relevance to our daily lives as
well. It's important to understand that although making sense of
mathematics on paper may seem difficult at first, it's worth pursuing
because developing this proficiency can help tremendously in class
discussions and exams while also providing insights into real-world
problems we might face on a regular basis.

\hypertarget{what-makes-writing-mathematics-different}{%
\subsection{What Makes Writing Mathematics
Different?}\label{what-makes-writing-mathematics-different}}

Writing mathematics is an artful combination of two distinct languages,
natural and mathematics. Natural language offers the writer numerous
expression possibilities but can be ambiguous. Whereas writing with
mathematical symbols requires clarity yet allows succinctness when
conveying complex concepts. Accurate language is a necessity when it
comes to mathematics. To properly communicate your thoughts and
opinions, ensure that each word carries the correct connotation you
intend for them to have.

\begin{quote}
The precision of words is essential in math-based conversations.
\end{quote}

To comprehend mathematically expressed ideas, readers must take their
time and read content several times while allowing themselves ample
pauses for contemplation along the way. Further, mathematics writing
often serves as reference material, which necessitates that its contents
should ideally be accessed piece by piece upon demand rather than
require deep immersion into text volumes at once.

Writing mathematics has rules -- some narrow, others broad. Small
conventions relate to sentence structure (including punctuation) and are
easily verifiable. Broader ones involve general style and strategies
that depend on the author's discretion. For a full primer on
mathematical writing see (\textcite{krantz2017primer}).

Crafting an effective document takes more than just following a few
simple rules; it requires mastering composition. This skillfulness is
where the art of writing comes in, with broad and deep strategies for
your entire work. It's all about connection: linking individual
sentences together for clarity and flow while adhering to specific
sentence structure regulations like commas or mathematical
terminologies. At its most complex level, these interrelated
requirements can create intricate webs that are both precise yet
engaging -- but will result in a masterpiece.

Now it's time to dive into each of \textbf{the five levels for writing
mathematics}: sentence, paragraph, section, chapter, and book.

\hypertarget{sentence-level}{%
\subsection{Sentence Level}\label{sentence-level}}

Breaking complex mathematical ideas into more accessible parts is
essential for effective communication. Using the right voice and symbols
in your writing is crucial for success.

\begin{quote}
These small details can make comprehension natural -- but mastering them
is fundamental when creating potent written works.
\end{quote}

Here are the ten rules that I use when writing mathematics.

\begin{enumerate}
\def\labelenumi{\arabic{enumi}.}
\tightlist
\item
  Do not start a sentence with a mathematical symbol.
\item
  Do not end a sentence with a mathematical symbol, instead use the
  correct punctuation.
\item
  Replace logical symbols with words, unless you are writting on logic.
\item
  Do not use a colon at the end of a sentence. Do not use sentence
  fragments, just use complete statements only.
\item
  Do spell out small numbers, when they are not being used
  mathematically.
\item
  Use ``we'' (``you'' and ``me'' together) for formal exposition, use
  ``I'' (``you'' and ``I'') for informal discussions.
\item
  Use ``that'' to indicate a specific object (a restrictive, essential
  clause) and not for making a point of inference.
\item
  Use ``which'' to add information to objects (a nonrestrictive,
  nonessential clause) and not as a conjunction.
\item
  Separate formulas with words. Do not list formulas, but rather
  communicate with a reader.
\item
  Write a sentence that flows logically from left to right to eliminate
  confusion.
\end{enumerate}

For more on these rules see (\textcite{knuth1989mathematical}).

\hypertarget{paragraph-level}{%
\subsection{Paragraph Level}\label{paragraph-level}}

Crafting the perfect opening paragraph is essential to grabbing a
reader's attention and inviting them on an engaging journey through your
prose. To do so, divide your work into linear segments to be presented
in a hierarchical development for ease of comprehension. Retain an even
flow by keeping notation terms familiar while avoiding lengthy
explanations or eloquent phrasing that could confuse readers.
Furthermore, provide previews of upcoming topics before delving deeper
into the content so they can prepare themselves mentally as you write.
Adding visuals where appropriate will also aid understanding -- use
examples and counterexamples along with suggestive references when
necessary to maximize clarity without sacrificing creativity.

\begin{enumerate}
\def\labelenumi{\arabic{enumi}.}
\tightlist
\item
  Begin a paragraph with your best sentence, you ensure that readers
  stay hooked.
\item
  Ensure that each sentence has its own distinct beat. Peruse your
  writing and fine-tune it until everything flows smoothly.
\item
  A theorem (definition, lemma, corollary, etc.) should stand firmly on
  its own two feet and not rely upon what came before.
\item
  Pay particular attention to terms like ``therefore'', as they are
  essential for achieving the right cadence and has many variations such
  as ``whence'',``hence'', ``and so'', etc.
\item
  Break up paragraphs by displaying \textbf{important formulas} on a
  line by themselves.
\item
  Don't over complicate things with unnecessary subscripts,
  superscripts, or other vertically spaced symbols.7. Break up long
  sentences into simple ones, and break up mathematics (but not
  important formulas) into readable text.
\item
  Don't be tempted to resort to the use of technical jargon without good
  reason.
\item
  Every paragraph should have a (mathematical) point, make it clear.
\item
  Write a paragraph that flows logically from sentence to sentence to
  eliminate confusion.
\end{enumerate}

Mathematics writing requires you to effectively communicate your thought
process and convince the reader that your solution is valid. A
first-rate mathematical exposition should provide precise explanations
while also being persuasive enough to satisfy a skeptical audience.

\hypertarget{section-level}{%
\subsection{Section Level}\label{section-level}}

Crafting a clear and concise section requires deliberate structure,
consistent information, and reader-minded readability.

\begin{enumerate}
\def\labelenumi{\arabic{enumi}.}
\item
  \textbf{Commmunicate}. While writing, the writer must envision what
  areas of confusion could arise for their reader and strive to ensure
  that these are addressed. This foresight is crucial in order to
  effectively communicate with readers; otherwise it risks creating
  misunderstandings instead of a meaningful exchange. To ensure accuracy
  and clarity, having an audience in mind while writing is not simply
  beneficial but necessary.
\item
  \textbf{Examples}. Mastering the art of example and counterexample can
  be a powerful tool. However, it is important not to underestimate its
  power -- examples should have some ``spark'' or element that will
  engage readers with thought-provoking insight they might otherwise
  miss. Additionally, don't forget the importance of providing context
  by using both examples and counterexamples which illustrate
  definitions/results while simultaneously clarifying any underlying
  essential assumptions.
\item
  \textbf{Figures}. Whenever possible, facilitate understanding by
  utilizing visual elements that have the power to communicate complex
  ideas quickly and accurately. Design figures simply with comprehensive
  captions as reinforcement for concepts discussed in textual form; this
  will help bring key points of proof or argumentation into clear focus
  without being bogged down in minutiae. Don't be afraid to utilize
  graphs over tables when illustrating main topics -- an illustration
  can often paint a far more detailed picture than words alone.
\item
  \textbf{Efficiency}. When writing proofs, strive for efficiency by
  taking advantage of earlier results and avoiding unnecessary
  repetition. If a current proof seems to be significantly similar to
  one already constructed in the past, think beyond simple alterations;
  try finding an interesting overarching generalization that ties both
  together. When crafting especially intricate arguments, provide
  strategic comments throughout your construction -- noting the purpose
  behind each step taken as well as providing overviews at appropriate
  sections can make lengthy demonstrations considerably easier to
  follow.
\item
  \textbf{Strategize}. When tackling a complex proof, take the time to
  strategize. Start by outlining how you plan to approach it and give
  periodic updates on your progress during the process. Demonstrate why
  particular steps are necessary and look for interesting similarities
  between results in order avoid reiterating similar arguments multiple
  times throughout. Exploring potential common generalizations of
  different elements can offer valuable insights too.
\end{enumerate}

Simply put, \textbf{be strategically efficient}.

Crafting a perfect section for your chapter is no easy feat. It requires
the careful placement of each element -- from preliminary discussions to
lemmas, proofs and more -- into an order that both makes sense logically
while also being effectively digested by readers. The ideal outcome
presents every detail in such away so as to capture attention and
surprise them with unexpected connections between elements they didn't
previously foresee; all parts fit together like pieces of a puzzle,
forming an explanation or argument far greater than its individual
components could ever were it not organized in this way.

\hypertarget{chapter-level}{%
\subsection{Chapter Level}\label{chapter-level}}

To effectively communicate your math, it's essential to take into
account the background knowledge of your intended audience. Don't let
unfamiliar terminology confuse them; provide additional explanation as
needed for clarity and comprehension. Consider adding appendixes when
discussing more complex or specialized topics that may require deeper
context in order to be fully understood by less experienced readers.

A clearly structured and properly ordered mathematics essay is far more
accessible than a disorganized one. Establishing an introduction that
introduces the important theorems, provides insight into why it matters,
and previews what's to come can ``hook'' readers in for further
exploration of your work. Making sure your chapter has these components
will help ensure its readability -- so put some thought into how you
want to order yours!

When creating math content, it is essential to keep the background and
expertise of your target audience in mind. Presenting too much detail
can be overwhelming for those without an expert's understanding;
however, avoiding explanations altogether risks leaving readers feeling
lost or confused. Consider breaking up complex material into separate
sections for a better grasp by all audiences as well as providing
additional helpful resources such as appendixes when applicable.

A well-crafted document requires a sound strategy. To achieve this,
start by outlining the expected sections in your paper or chapter so you
know exactly what to include in each step of its development. Establish
necessary definitions that will inform and structure every result and
proof before testing them for accuracy with lemmas; dabbling examples
along the way allows readers to apply concepts practically. Repeat these
steps until there is an obvious logical flow from beginning to end, then
add interstitial comments as needed -- which may provide further context
on certain arguments, compare opposing points within literature topics
or simply warn against any potential missteps throughout your work.

\hypertarget{book-level}{%
\subsection{Book Level}\label{book-level}}

Make sure to keep your reader informed throughout; start each segment
with a brief introduction and perhaps an outline of what's ahead. Avoid
lengthy stretches without any clear direction, revealing key conclusions
only at the end. Instead, provide hints as you progress by stating
outcomes such as ``It follows that \ldots{}''. Finally, be sure to
reiterate essential points in conclusion for easy recall later on.

When composing equations and propositions, don't make it difficult for
readers to find the references; clearly number each item along with
providing a content/name descriptor. To ensure concentration isn't
broken by excessive page flipping, repeat any simple math expressions
used in addition to reminding the reader of earlier analysis or unusual
notations they may have encountered. Don't be afraid to reiterate
information if necessary -- however remain mindful that too much
repetition can become exhausting.

\hypertarget{and-beyond}{%
\subsection{And Beyond}\label{and-beyond}}

Mathematics is truly a language of the mind, and at its heart lies
\textbf{mathemas} -- an ancient Greek word encapsulating knowledge,
cognition, understanding and perception. In college-level math courses
this universal tongue grows in complexity; equations alone are no longer
sufficient to describe it as ideas become so intricate they must be
expressed through sentences and paragraphs. Thus one can see that
mathematics moves beyond mere numbers or calculations: It is about
uncovering new ways of interpreting the universe around us.

\begin{quote}
Do not give up -- mastering mathematics can open a world of
opportunities.
\end{quote}

While it might be hard to get started, writing mathematics is an
invaluable skill that requires patience and dedication. Developing this
capability takes practice and experience, but if you put in the work
your mastery of mathematical exposition will stay with you for life.


\printbibliography



\thispagestyle{empty}


\end{document}
